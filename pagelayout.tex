\chapter{Layout}

\section{Justification and alignment}

\LaTeX{} justifies text remarkably well.
Instead of arranging it one line at a time---like most word processors,
web browsers, and e-readers do---it considers every possible line break
in a paragraph, then picks the ones that give the best overall
spacing.\punckern\endnote{Donald E.~Knuth and Michael F.~Plass,
\textit{Breaking Paragraphs Into Lines} (Stanford, 1981)}
Combined with automatic hyphenation,
which permits line breaks in the middle of words,\punckern\endnote{%
Franklin Mark Liang,
\textit{Word Hy-phen-a-tion by Com-put-er} (Stanford, 1983),
\http{www.tug.org/docs/liang/}}
it can produce better paragraph layouts than almost any other software.

But sometimes we don't want justified text.
If you would like it to be flush left,
place it in a \texttt{flushleft} environment
or add \verb|\raggedright| to the current group.
To center it, place it in a
\texttt{center} environment or add \verb|\centering| to the current group.
And to flush it against the right margin,
use a \texttt{flushright} environment or \verb|\raggedleft|.

\begin{leftfigure}
\begin{lstlisting}
\begin{flushleft}
This text is flush left, with a ragged right edge.
Some prefer this layout since the space between words
is more consistent than it is in justified text.
\end{flushleft}

\begin{center}
This text is centered.
\end{center}

\begin{flushright}
And this text is flush right.
\end{flushright}
\end{lstlisting}
\end{leftfigure}
sets
\begin{flushleftfigure}
\begin{minipage}{0.99\textwidth}
\lm%
\begin{flushleft}
This text is flush left, with a ragged right edge.
Some prefer this layout since the space between words
is more consistent than it is in justified text.
\end{flushleft}

\begin{center}
This text is centered.
\end{center}

\begin{flushright}
And this text is flush right.
\end{flushright}
\end{minipage}
\end{flushleftfigure}

\section{Lists}

\LaTeX{} provides several environments for creating lists:
\texttt{itemize}, \texttt{enumerate}, and \texttt{description}.
In all three, each item starts with an \verb|\item| command.
To create bulleted lists, use the \texttt{itemize} environment.
With:
\begin{leftfigure}
\begin{lstlisting}
\begin{itemize}
\item 5.56 millimeter
\item 9 millimeter
\item 7.62 millimeter
\end{itemize}
\end{lstlisting}
\end{leftfigure}
you get
\begin{leftfigure}
\lm%
\begin{itemize}[leftmargin=*]
\item 5.56 millimeter
\item 9 millimeter
\item 7.62 millimeter
\end{itemize}
\end{leftfigure}

\bigskip
\noindent \texttt{enumerate} numbers its lists:
\begin{leftfigure}
\begin{lstlisting}
\begin{enumerate}
\item Collect underpants
\item ?
\item Profit
\end{enumerate}
\end{lstlisting}
\end{leftfigure}
becomes
\begin{leftfigure}
\lm%
\begin{enumerate}[leftmargin=*]
\item Collect underpants
\item ?
\item Profit
\end{enumerate}
\end{leftfigure}

\bigskip
\noindent The \texttt{description} environment starts each item with some
emphasized label,
then indents subsequent lines for that item:
\begin{leftfigure}
\begin{lstlisting}
\begin{description}
\item[Alan Turing] was a British mathematician who
    laid much of the groundwork for computer science.
    He is perhaps most remembered for his model of
    computation, the Turing machine.
\item[Edsger Dijkstra] was a Dutch computer scientist.
    His contributions in many subdomains---such as
    concurrency and graph theory---are still in wide use.
\item[Leslie Lamport] is an American computer scientist.
    He defined the concept of sequential consistency,
    which is used to safely communicate between tasks
    running in parallel.
\end{description}
\end{lstlisting}
\end{leftfigure}
gives us
\begin{leftfigure}
\lm%
\begin{description}
\item[\lm Alan Turing] was a British mathematician who
    laid much of the groundwork for computer science.
    He is perhaps most remembered for his model of computation,
    the Turing machine.
\item[\lm Edsger Dijkstra] was a Dutch computer scientist.
    His contributions in many subdomains---such as
    concurrency and graph theory---are still in wide use.
\item[\lm Leslie Lamport] is an American computer scientist.
    He defined the concept of sequential consistency,
    which is used to safely communicate between tasks
    running in parallel.
\end{description}
\end{leftfigure}

\section{Columns}

We often split pages into several columns, especially when printing on
\textsc{a4} or \acronym{us}~\textsc{l}etter paper,
since it provides more comfortable line widths with standard
8--12\,pt text sizes.\punckern\footnote{You'll see different advice depending
on where you look, but as a rule of thumb,
aim for 45 to 80 characters (including spaces) per line.
If a line is too long, readers have an uncomfortable time scanning for
the start of the next one.
If a line is too short, it doesn't have much inter-word spacing to adjust,
which can lead to odd gaps or excessive hyphenation.}
You can either add the \texttt{twocolumn} option to your document class,
which splits everything in two, or you can use the \texttt{multicols}
environment from the \texttt{multicol} package:
\begin{leftfigure}
\begin{lstlisting}
One nice feature of the \texttt{multicol} package
is that you can combine arbitrary layouts.
\begin{multicols}{2}
This example starts with one column,
then sets the following section as two.
The \texttt{multicols} environment splits the text
inside it so that each column is about the same height.
\end{multicols}
\end{lstlisting}
\end{leftfigure}
arranges
\begin{leftfigure}
\lm%
One nice feature of the {\lt multicol} package
is that you can combine arbitrary layouts.
\begin{multicols}{2}
This example starts with one column,
then sets the following section as two.
The \mbox{\lt multicols} environment splits the text
inside it so that each column is about the same height.
\end{multicols}
\end{leftfigure}

\section{Page breaks}

Some commands, like \verb|\chapter|, insert page breaks.
You can add your own with \verb|\clearpage|.
When using the \texttt{twoside} document class option for double-sided
printing, you can break to the front of the next page with
\verb|\cleardoublepage|.

\section{Footnotes}

Footnotes are useful for references,
or for remarks that readers might find helpful
but aren't crucial to the main text.
The \verb|\footnote| command places a marker at its location in the
body text, then sets its argument at the bottom of the current
page:
\begin{leftfigure}
\begin{lstlisting}
I love footnotes!\footnote{Perhaps a bit too much\dots}
\end{lstlisting}
\end{leftfigure}
proclaims
\begin{leftfigure}
\lm%
I love footnotes!\footnote{Perhaps a bit too much\dots}
\end{leftfigure}

\section{What next?}
\begin{itemize}
\item Control paragraph spacing with KOMA~Script or the \texttt{parskip} package.
\item Set the page size and margins with the \texttt{geometry} package.
\item Customize list formatting with the \texttt{enumitem} package.
\item Create tables with the \texttt{tabular} environment.
\item Align text with tab stops using the \texttt{tabbing} environment.
\item Customize footnote symbols and layout with KOMA~Script and the
    \texttt{footmisc} package.
\item Insert horizontal and vertical space with commands like
    \verb|\vspace|, \verb|\hspace|, \verb|\vfill|, and \verb|\hfill|.
\item Learn what units \LaTeX{} provides for specifying spacing. \\
    (We've already mentioned a few here, like
    \texttt{pt}, \texttt{bp}, and \texttt{in}.)
\end{itemize}
