% Build this document with LuaTeX, a modern Unicode-aware LaTeX engine
% that uses system TTF and OTF font files.
% This is needed for the fontspec, microtype, and nolig packages.
%
% We're using KOMA Script to hand-tune footnotes and TOC appearance.
% It should be available in your texlive distribution,
% which is how most distros package LaTeX.
\documentclass[fontsize=11bp, numbers=endperiod, draft=true]{scrbook}

% Margins: see http://practicaltypography.com/page-margins.html and
% http://practicaltypography.com/line-length.html
% We're aiming for 80-ish characters per line.
\usepackage[a5paper,
            inner=0.65in,outer=0.55in,top=0.75in,bottom=0.5in,
            footnotesep=11bp,
            footskip=3em,
            includefoot,
           ]{geometry}

\usepackage[fleqn]{amsmath}

\usepackage{fontspec}

\setmainfont[
    Ligatures=TeX,
    Numbers={Proportional,Lowercase},
    % Coefficients for font-provided space, stretch, and shrink.
    % Garamond Premier is a bit tighter than Adobe Garamond,
    % but stretching it out about 20% gives about the same metrics.
    WordSpace={1.2,1.2,1},
    UprightFeatures={
        SizeFeatures={
            {Size={-10},Font=garamondpremrpro-capt},
            {Size={10-15},Font=garamondpremrpro},
            {Size={15-23},Font=garamondpremrpro-subh},
            {Size={23-},Font=garamondpremrpro-disp}
        },
    },
    BoldFeatures={
        SizeFeatures={
            {Size={-10},Font=garamondpremrpro-smbdcapt},
            {Size={10-15},Font=garamondpremrpro-smbd},
            {Size={15-23},Font=garamondpremrpro-smbdsubh},
            {Size={23-},Font=garamondpremrpro-smbddisp}
        },
    },
    ItalicFeatures={
        SizeFeatures={
            {Size={-10},Font=garamondpremrpro-itcapt},
            {Size={10-15},Font=garamondpremrpro-it},
            {Size={15-23},Font=garamondpremrpro-itsubh},
            {Size={23-},Font=garamondpremrpro-itdisp}
            },
    },
    BoldItalicFeatures={
        SizeFeatures={
            {Size={-10},Font=garamondpremrpro-smbditcapt},
            {Size={10-15},Font=garamondpremrpro-smbdit},
            {Size={15-23},Font=garamondpremrpro-smbditsubh},
            {Size={23-},Font=garamondpremrpro-smbditdisp}
        },
    },
]{Garamond Premier Pro}

\setsansfont[Ligatures=TeX,
             Scale=MatchUppercase,
             Style=Alternate, % Straight-legged R
             UprightFont = *-55Rg,
             ItalicFont = *-56It,
             BoldFont = *-65Md,
             BoldItalicFont = *-66MdIt
             ]{NHaasGroteskTXPro}

\setmonofont[
    Numbers=SlashedZero,
    Scale=MatchLowercase,
    UprightFeatures={
        SizeFeatures={
            {Size={-10}, Font=DriveMono-Book},
            {Size={10-}, Font=DriveMono-Regular},
        },
    },
    ItalicFeatures={
        SizeFeatures={
            {Size={-10}, Font=DriveMono-BookItalic},
            {Size={10-}, Font=DriveMono-Italic},
        },
    },
    BoldFont=DriveMono-Bold,
    BoldItalicFont=DriveMono-BoldItalic
]{Drive Mono}

\setmathrm{Latin Modern Roman}

% We'll be using this quite a bit:
\newfontfamily{\lm}[
    Ligatures=TeX,
    SmallCapsFont = * Caps
]{Latin Modern Roman}
\newfontfamily{\lt}{Latin Modern Mono}

\usepackage{polyglossia}
\setdefaultlanguage[variant=american]{english}
\setotherlanguage{french}

\usepackage{microtype} % Font expansion, protrusion, and other goodness

% Disable ligatures across grapheme boundaries
% (see the package manual for details.)
\usepackage[american]{selnolig}
% I have no idea why Garamond has this on by default.
\nolig{Th}{T|h}

% Use symbols for footnotes, resetting each page
\usepackage[perpage,bottom,symbol*]{footmisc}

% Left flush footnotes. See the KOMA Script manual.
\deffootnote[1em]{1em}{1em}{\thefootnotemark}
% Set the width of the rule separating body text and footnotes
\setfootnoterule{0.8\textwidth}

% Like many fonts, Equity's asterisk is already set in a "superscripted" form.
% Superscripting *that* makes it annoyingly small.
% To fix this, we have to redefine footnote marks so that they aren't superscript,
% then raise all the other symbols.
%
% Feel free to remove this if your body type doesn't have this peculiarity,
% but unfortunately many do.
% See http://tex.stackexchange.com/a/16241
%
% We use the Unicode symbols themselves (instead of \dagger, \ddagger, \P, etc.)
% because the latter fall back to Computer Modern/Latin Modern in some cases,
% (e.g., if you're using mathastext instead of unicode-math).
% Alternatively, you could use \textdagger, \textddagger, etc.,
% but this seems more concise.
\DefineFNsymbols*{tweaked}{%
    {*}%
    {\textsuperscript†}%
    {\textsuperscript‡}%
    {\textsuperscript{◊}}%
    {\textsuperscript{§}}%
    {**}%
    {\textsuperscript{††}}%
    {\textsuperscript{‡‡}}%
}
\setfnsymbol{tweaked}
\deffootnotemark{\thefootnotemark}

\DeclareTOCStyleEntry[%
    beforeskip=8bp,
    entryformat = \bfseries,
    entrynumberformat = \addfontfeature{Numbers={Tabular,Uppercase}},
    pagenumberformat = \addfontfeature{Numbers={Tabular,Uppercase}},
    linefill = \TOCLineLeaderFill
]{tocline}{chapter}
\DeclareTOCStyleEntry[%
    beforeskip=1bp,
    entrynumberformat = \addfontfeature{Numbers={Tabular,Uppercase}},
    pagenumberformat = \addfontfeature{Numbers={Tabular,Uppercase}},
    indent=0.4in
]{tocline}{section}
\DeclareTOCStyleEntry[%
    beforeskip=0pt,
    entrynumberformat = \addfontfeature{Numbers={Tabular,Uppercase}},
    pagenumberformat = \addfontfeature{Numbers={Tabular,Uppercase}},
    indent=0.6in
]{tocline}{subsection}

% Don't use a sans font for description labels.
\addtokomafont{descriptionlabel}{\rmfamily}
% Sections and such use serif type.
\setkomafont{disposition}{\rmfamily}
% Uppercase numbers for chapters.
\addtokomafont{chapter}{\addfontfeature{Numbers=Uppercase}}
% Set size and style of section and subsections
\setkomafont{section}{\Large\itshape}
\setkomafont{subsection}{\large\itshape}
% Don't put big spaces before chapter headings
\renewcommand{\chapterheadstartvskip}{\vspace{0.15in}}

\setcapwidth[c]{.75\textwidth}
%\setcapmargin{0pt}
\setkomafont{caption}{\sffamily\footnotesize}
\setkomafont{captionlabel}{\sffamily\footnotesize}
\renewcommand*{\figureformat}{}
\renewcommand*{\tableformat}{}
\renewcommand*{\captionformat}{}

% Use uppercase numbers for numbered lists.
% (We're using lowercase ones for the body text.)
% See http://tex.stackexchange.com/a/133186
\usepackage{enumitem}
\setlist[enumerate]{font=\addfontfeatures{Numbers=Uppercase}}
\setlist[description]{leftmargin=1em}

\usepackage{tabularx}

% Custom footer
\usepackage[draft=false]{scrlayer-scrpage}
\clearpairofpagestyles
\pagestyle{scrheadings}
\setkomafont{pagefoot}{\upshape}
\lefoot*{\thepage}
\rofoot*{\thepage}

\usepackage{changepage} % For adjustwidth

\usepackage{mflogo} % for METAFONT
\usepackage{metalogo} % for \LuaLaTeX

\usepackage{multicol}

\usepackage{endnotes}
% OTF goodness...
\renewcommand\makeenmark{{\addfontfeature{VerticalPosition=Superior}\theenmark}}

\usepackage{listings}
\lstset
{
    language=[LaTeX]TeX,
    breaklines=false,
    basicstyle=\ttfamily,
    keywordstyle=\ttfamily,
    commentstyle=\ttfamily,
}

% Indent code examples, etc., by double the text size.
\newenvironment{leftfigure}
  {\par\vspace{0.5\baselineskip minus 0.3\baselineskip}\begin{adjustwidth}{22bp}{0pt}}
  {\end{adjustwidth}\vspace{0.5\baselineskip minus 0.3\baselineskip}}

% Like the above, but with no adjustwidth
\newenvironment{flushleftfigure}
  {\par\vspace{0.5\baselineskip minus 0.3\baselineskip}\noindent\ignorespacesafterend}
  {\vspace{0.5\baselineskip minus 0.3\baselineskip}\par\noindent\ignorespacesafterend}

\newenvironment{centerfigure}
  {\par\vspace{0.5\baselineskip minus 0.3\baselineskip}\begin{adjustwidth}{22bp}{22bp}\centering}
  {\end{adjustwidth}\vspace{0.5\baselineskip minus 0.3\baselineskip}}

\usepackage{graphicx}

\usepackage{csquotes}

\title{Modern \texorpdfstring{\LaTeX}{LaTeX}}
\author{Matt Kline}
\date{\today}

% Custom footer
% Hyperlinks
\usepackage[unicode,pdfusetitle,hidelinks]{hyperref}

% Use \punckern to overlap periods, commas, and footnote markers
% for a tighter look.
% Care should be taken to not make it too tight - f" and the like can overlap
% if you're not careful.
\newcommand{\punckern}{\kern-0.2ex}
% For placing commas close to, or under, quotes they follow.
% We're programmers, and we blatantly disregard American typographical norms
% to put the quotes inside, but we can at least make it look a bit nicer.
\newcommand{\quotekern}{\kern-0.5ex}


% Create an unbreakable string of text in a monospaced font.
% Useful for `command --line --args`
\newcommand{\monobox}[1]{\mbox{\texttt{#1}}}

\newcommand{\otffrac}[2]{\mbox{%
    {\addfontfeature{VerticalPosition=Superior}#1}%
    ^^^^2044% Unicode fraction slash
    {\addfontfeature{VerticalPosition=Inferior}#2}%
}}
\newcommand{\otford}[2]{\mbox{%
    {\addfontfeature{Numbers=LowercaseOff}#1}%
    {\addfontfeature{VerticalPosition=Superior}#2}%
}}

% C++ looks nicer if the ++ is in a monospace font and raised a bit.
% Also, use uppercase numbers to match the capital C.
\newcommand{\plusplus}{\raisebox{0.1ex}{++}}
\newcommand{\cpp}[1]{C\kern-0.1ex\plusplus{\addfontfeature{Numbers=LowercaseOff}#1}}

% Italicize new terms
\newcommand{\introduce}[1]{\textit{#1}}

% Letterspace acronyms a bit.
\newcommand{\acronym}[1]{\textsc{\addfontfeature{LetterSpace=5}#1}}

% "Chapter <num>" references
\newcommand{\chapref}[1]{chapter~\ref{#1}}

% monospace URLs (without setting the http://...)
\newcommand{\http}[1]{\href{http://#1}{\texttt{#1}}}
\newcommand{\https}[1]{\href{https://#1}{\texttt{#1}}}

\newcommand{\edition}{Second edition \acronym{(wip)}}

% See http://tex.stackexchange.com/a/68310
\makeatletter
\let\runauthor\@author
\let\rundate\@date
\let\runtitle\@title
\makeatother

% Spend a bit more time to get better word spacing.
% See http://tex.stackexchange.com/a/52855/92465
\emergencystretch=1ex

% Do as I say, not as I do.
\widowpenalty=10000
\clubpenalty=10000

\begin{document}
\fontsize{11bp}{13bp}\selectfont

\frontmatter
\setcounter{secnumdepth}{0}
\setlength\parindent{0pt}

% Custom title instead of \maketitle
\pagenumbering{gobble}
\vspace*{1in}
\begin{center}
\fontsize{0.5in}{0.7in}\selectfont
Modern

\fontsize{1in}{0.9in}\selectfont
\LaTeX

\normalsize
\vspace{1.5\baselineskip}
\edition
\vspace{2in}

\LARGE
\runauthor
\end{center}
\clearpage

{\raggedright%For the page
\null
\vfill
{\addfontfeature{Numbers={Proportional,Uppercase}}
Copyright © 2018--2019 \\
by \runauthor
\bigskip

This book is licensed under the \\
Creative Commons Attribution-ShareAlike~4.0 International License. \\
In short, you are free to share, translate, adapt, or improve this book
so long as you give proper credit and provide your contributions under
the same license. \\
The license's full text is available at \\
\https{creativecommons.org/licenses/by-sa/4.0/legalcode}
}
\vfill

The author apologizes for any typos,
f\raisebox{-0.1ex}{o}rmatt\raisebox{0.1ex}{i}ng mistakes,
inaccuracies,
and other flubs.
He welcomes you to point them out via this book's Git repository at \\
\https{github.com/mrkline/latex-book}

\vspace{\baselineskip}
Questions, comments, concerns, and diatribes can also be emailed to \\
\texttt{matt <at> bitbashing.io}

\vspace{\baselineskip}
The author does not have a checking account with the Bank of San Serriffe,
but will happily compensate your feedback with a beverage of your choice
next time we meet.

\vspace{0.5in}
\edition, typeset \today.
} % end ragged right
\clearpage

\vspace*{1in}
{\itshape%
To Max, who once told me about a cool program he used to type up
his college papers.
}
\cleardoublepage

\pagenumbering{roman}
\tableofcontents

\mainmatter
% Indent by one lead, as suggested in The Elements of Typographic Style.
\setlength\parindent{14bp}

\pagenumbering{arabic}
\setcounter{page}{1} % Restart page numbering after the ToC.
\cleardoublepage

\chapter{Typography and You}
\label{typography}

Life is a parade of written language.
Wherever you go,
ads, apps, articles, essays, emails, and messages
shove text in your face.
But when you read that text, you see so much more than the author's
verbiage.
Consciously or not, you notice the shapes and sizes of letters.
You notice how those letters are arranged into words,
how those words are arranged into paragraphs,
how those paragraphs are arranged onto pages and screens.
You notice \introduce{typography}.
% Smoke test: a line should be 2-3 alphabets wide
%\\ abcdefghijklmnopqrstuvwxyzabcdefghijklmnopqrstuvwxyzabcdefghijklmnopqrstuvwxyz
\begin{leftfigure}
\fontspec{TeX Gyre Termes}\fontsize{12bp}{24bp}\selectfont\raggedright
Typography is why these lines evoke memories of awful essays
you wrote in school.
Do many books look this way? Why not?
\end{leftfigure}
\medskip
\noindent There is a reason street signs don't look like this:
\begin{leftfigure}
\fontspec[Scale=MatchUppercase]{KJV1611}\Large E Gorham St.
\end{leftfigure}
And why a very important switch in a spaceship is labeled like this:
\begin{leftfigure}
\fontspec[Scale=MatchUppercase]{Futura-Med}CM/SM SEP
\end{leftfigure}
Not like this:
\begin{leftfigure}
\fontspec{Pinyon Script}\Large CM/SM Sep
\end{leftfigure}

Effective written communication is about the shapes and layout of words,
not just which ones you choose.
Good typography isn't just art---it's function.
And if you care about any of this,
you should try \LaTeX,\punckern\footnote{Pronounced ``lay-tech''
or ``lah-tech''}
a program for crafting written documents.
By carefully handling subtle details,
it produces beautiful typography with little effort.
Modern versions can also leverage recent\footnote{By recent,
I mean ``from the mid-1990s''\quotekern, but web browsers and desktop publishing
software are only just starting to catch up.} advances in digital typesetting,
offering you the same tools used by professional graphic designers and
publishers.

\section{\texorpdfstring{\LaTeX}{LaTeX}?}

\LaTeX{} is an alternative to ``word processors'' like
Microsoft Word, Apple Pages, Google Docs,
and LibreOffice Writer.
These other applications operate on the principle of
\introduce{What You See Is What You Get}
\acronym{(wysiwyg)}, where what's on screen is the same
as what comes out of your printer.
\LaTeX{} is different. Here, documents are written as
``plain'' text files, using \introduce{markup}
to specify how the final result should look.
If you've done any web design, this is a similar
process---just as \acronym{html} and \acronym{css}
describe the page you want browsers to draw,
markup describes the appearance of your document to \LaTeX.

\begin{samepage}
\begin{leftfigure}
\begin{lstlisting}
\LaTeX{} is an alternative to ``word processors'' like
Microsoft Word, Apple Pages, Google Docs,
and LibreOffice Writer.
These other applications operate on the principle of
\introduce{What You See Is What You Get}
\acronym{(wysiwyg)}, where what's on screen is the same
as what comes out of your printer.
\LaTeX{} is different. Here, documents are written as
``plain'' text files, using \introduce{markup}
to specify how the final result should look.
If you've done any web design, this is a similar
process---just as \acronym{html} and \acronym{css}
describe the page you want browsers to draw,
markup describes the appearance of your document to \LaTeX.
\end{lstlisting}
\captionof{figure}{The \LaTeX{} markup for the paragraph above}
\end{leftfigure}
\end{samepage}

This might seem strange if you haven't worked with markup before,
but it comes with a few advantages:
\begin{enumerate}
\item You can handle your writing's content and its presentation separately.
    At the start of each document,
    you describe the design you want.
    \LaTeX{} takes it from there, consistently formatting your whole text
    the way you asked.
    Compare this to a \acronym{wysiwyg} system,
    where you constantly deal with appearances
    as you write.
    If you changed the look of a caption,
    were you sure to find all the other captions and do the
    same?
    If the program formats something in a way you don't like,
    is it hard to fix?%\footnote{I spent far too much of my childhood
    %fighting with Word about how it wrapped text around images.}

\item You can define your own commands, then tweak them to instantly adjust
    every place they're used.
    For example, the \verb|\introduce| and \verb|\acronym| commands
    from the example paragraph are my own creations.
    The former \introduce{italicizes} text, and the latter sets words in
    \acronym{small caps} with a bit of extra
    \mbox{\textsc{\addfontfeature{LetterSpace=15}letterspacing}} so the characters
    don't look \textsc{\addfontfeature{LetterSpace=-10}too crowded}.
    If I decide that I'd rather introduce new terms
    \textbf{\itshape with this look}, or that acronyms should be displayed
    {\small\addfontfeature{LetterSpace=6} LIKE THIS},
    I just change the two lines that define those commands.
    Everywhere they're used immediately takes on the new look.

\item Being able to save the document as plain text has its own benefits:
    \begin{itemize}
    \item It can be read and understood with any text editor.
    \item Structure is immediately visible
        and simple to replicate.\punckern\footnote{Compare this to
        \acronym{wysiwyg} systems, where it's not always obvious
        how certain formatting was produced or how to match it.}
    \item Content is easily generated by scripts and programs.
    \item Changes can be tracked with standard version control software,
        like Git or Mercurial.
    \end{itemize}
\end{enumerate}

\section{Another guide?}

You might wonder why the world needs another guide for \LaTeX{}.
After all, it's been around for decades.
A quick Amazon search finds nearly a dozen books on the topic.
There are plenty of resources online.

Unfortunately, most \LaTeX{} guides have two fatal flaws:
they are long, and they are old.
Beginners don't want---or need---hundreds of pages just to learn the basics,
and older guides waste your time with outdated information.
When \LaTeX{} was first released in 1986, none of the publishing technologies
we use today existed.
Adobe wouldn't debut their Portable Document Format for seven more years,
and desktop publishing was a fledgling curiosity.
This shows---badly---in many \LaTeX{} guides.
If you look for instructions to change your document's font,
you'll get swamped with bespoke nonsense.\punckern\footnote{%
Take the criticisms here with a grain of
salt. After all, the fact that all of the technology around \LaTeX{} became
obsolete---multiple times---is a testament to its staying power.}

The good news is that  \LaTeX{} has improved by leaps and bounds in recent years.
It's time for a guide that doesn't weigh you down with decades of legacy
or try (in vain) to be a comprehensive reference.
After all, you're a smart, resourceful person who knows how to use a
search engine.
This book will:

\begin{enumerate}
\item Teach you the fundamentals of \LaTeX.
\item Point you to places where you can learn more.
\item Show you how to take advantage of modern typesetting technologies.
\item End promptly thereafter.
\end{enumerate}
\vspace{\baselineskip}

\noindent Let's begin.

\chapter{Installation}
\label{installation}

You install \LaTeX{} on your computer as a \introduce{distribution}.
It comes with:
\begin{enumerate}
\item \LaTeX, the program---the thing that typesets text files into
    documents.\footnote{Well, actually, multiple \LaTeX{} programs,
    but we're getting to that.}
\item A common set of \LaTeX{} \introduce{packages}.
    Packages are bundles of code that do all sorts of things,
    like provide new commands or change a document's style.
    We'll see lots of them in action throughout this book.
\item Miscellaneous tools, like editors.
\end{enumerate}
Each major operating system has its own \LaTeX{} distribution:
\begin{description}
\item[Mac OS] has Mac\TeX. Grab it from \http{www.tug.org/mactex}
    and install it using the instructions there.

\item[Windows] has Mik\TeX.
    Install it from \https{miktex.org/download}.
    Mik\TeX{} has the helpful ability to automatically download
    additional packages as your documents use them for the first time.

\item[Linux and BSD] use \TeX{} Live.
    Like most software, it is provided through your
    \acronym{os}'s package manager.
    Linux distributions usually contain a \texttt{texlive-\allowbreak full}
    or \texttt{texlive-\allowbreak most} package that installs everything
    you need.\punckern\footnote{%
    If you'd rather keep the install size down,
    Linux distributions usually break \TeX{} Live into multiple distro packages.
    Look for ones with names like
    \texttt{texlive-\allowbreak core}, \texttt{texlive-\allowbreak luatex}
    and \texttt{texlive-\allowbreak xetex}.
    As you work with \LaTeX, you may need less-common packages,
    which usually have names like \texttt{texlive-\allowbreak latexextra},
    \texttt{texlive-\allowbreak science}, and so on.
    Of course, all of this may vary from one Linux distribution to another.}
\end{description}

\section{Editors}

Since \LaTeX{} source files are regular text files,
you write them with the usual choices: Vim, Emacs,
Sublime, VS~Code, and so on.\punckern\footnote{If you've never used
any of these, try a few.
They're popular with programmers and other folks who shuffle text around
screens all day. Just don't use Notepad. Life is too short.}
There are also editors designed specifically for \LaTeX{},
which often come with their own built-in \acronym{pdf} viewer.
(You can find a fairly comprehensive list on the \LaTeX{} Wikibook,
in its installation chapter. See Appendix~\ref{resources}.)

\section{Online options}

If you can't be bothered to install \LaTeX{} on your computer,
try online editors like Share\LaTeX{} or Overleaf.
This book won't focus on these web-based tools,
but the same basics apply.
Of course, you have less control over certain aspects
like available fonts, the version of \LaTeX{} that's used, and so on.

\chapter{Hello, \texorpdfstring{\LaTeX}{LaTeX}!}
\label{hello}

Now that you have \LaTeX{} installed,
let's try it out.
Open up your favorite text editor and save the following as \texttt{hello.tex}:
\begin{leftfigure}
\begin{lstlisting}
\documentclass{article}
% Say hello
\begin{document}
Hello, World!
\end{document}
\end{lstlisting}
\end{leftfigure}
Next, we'll run this file through \LaTeX{} (the program)\footnote{Not to be
confused with \LaTeX{} the lunchbox, \LaTeX{} the breakfast cereal,
or \LaTeX{} the flamethrower. The kids love this stuff!}
to get our document.
The installation placed several different versions---or
\introduce{engines}---on your machine,
but throughout this book, we'll use the newest ones:
\LuaLaTeX{} and \XeLaTeX.\punckern\footnote{See Appendix~\ref{history} for a
comparison of the various \LaTeX{} engines.}

If you are using a \LaTeX{}-specific editor, it should have some menu
to select the engine you'd like to use,
along with a button to generate your document.
Otherwise, run the following from your terminal:\footnote{%
How to work a terminal emulator,
make sure the newly-installed \LaTeX{} programs are in your \texttt{PATH},
and so on are all outside the scope of this book.
As is tradition, the leading dollar sign in this example just denotes a console
prompt, and shouldn't actually be typed.}
\begin{leftfigure}
\begin{lstlisting}
$ xelatex hello.tex
\end{lstlisting}
\end{leftfigure}
Feel free to try \texttt{lualatex} instead---there are a few differences
between the two that we'll discuss later, but either is fine for now.
With luck, you should see some output that ends in a message like:
\begin{leftfigure}
\begin{lstlisting}
Output written on hello.pdf (1 page).
Transcript written on hello.log.
\end{lstlisting}
\end{leftfigure}
And in your current directory, you should find a newly minted \texttt{hello.pdf}.
Open it up and you should see a page with this at the top:
\begin{leftfigure}
\lm Hello, World!
\end{leftfigure}
Congrats,
you created your first document!
Let's unpack what we just did.

All \LaTeX{} documents begin with a \verb|\documentclass| declaration,
which picks a base ``style'' to use.
Many classes are available---and you can even create your own---but common
ones include \texttt{article}, \texttt{report}, \texttt{book},
and \texttt{beamer}.\punckern\footnote{This last one is for slideshows.
The name is a German term for a projector.}
For the average document,
\texttt{article} is probably a good choice.
The next line, \verb|% Say hello|,
is a \introduce{comment}.
\LaTeX{} ignores the rest of a line once it sees a percent sign,
so we use it to leave notes for anybody reading
the document's source.\punckern\footnote{Including, perhaps most importantly,
a confused version of your future self!}
Finally, \verb|\begin{document}| tells \LaTeX{} that what follows
is our actual contents,
and \verb|\end{document}| states that we are finished.

Let's cover some more basics.

\section{Spacing}

\LaTeX{} generally handles inter-word spacing for you, regardless of how many
times you mash the space bar or tab key.
For example,
\begin{leftfigure}
\begin{lstlisting}
The number  of   spaces    between words doesn't   matter.
The same is true for space between sentences.

An empty line ends the previous paragraph and
starts the next.
\end{lstlisting}
\end{leftfigure}
yields
\begin{leftfigure}
\lm The number  of   spaces    between words doesn't   matter.
The same is true for space between sentences.

An empty line ends the previous paragraph and
starts the next.
\end{leftfigure}
Notice that \LaTeX{} automatically follows typographic
conventions, such as indenting new paragraphs and leaving more space after a
period than between words.
One quirk to be aware of is that comments ``eat'' any leading
space on the following line, so
\begin{leftfigure}
\begin{lstlisting}
This% weird, right?
  is strange.
\end{lstlisting}
\end{leftfigure}
gives
\begin{leftfigure}
\lm This% weird, right?
  is strange.
\end{leftfigure}

\section{Commands}

\LaTeX{} provides various commands to format your text,
and you can define your own as well.
Their names always begin with a backslash (\,\texttt{\textbackslash}\,),
contain only letters, and are case-sensitive.\punckern\footnote{%
% \verb doesn't play nicely in footnotes, so...
\texttt{\textbackslash foo}
is different from \texttt{\textbackslash Foo}, for example.}
Some commands need more information, or \introduce{arguments}:
\verb|\documentclass|,
for example, needs to know which class we want.
Arguments are enclosed in consecutive pairs of braces,
so if some command needed two arguments, we would type:
\begin{leftfigure}
\begin{lstlisting}
\somecommand{argument1}{argument2}
\end{lstlisting}
\end{leftfigure}

Many commands also take optional arguments.
They precede the mandatory ones,
are enclosed in square brackets, and are separated by commas.
Say you want to inform \LaTeX{} that your document should be printed as double-sided
pages\footnote{\texttt{twoside} introduces commands
that only make sense in the context of double-sided printing,
such as ones that skip to the start of the next odd page.
It also allows you to have different margins for even and odd pages,
which is useful for texts like this book.}
in 11~point
type.\punckern\footnote{We'll discuss font sizes in \chapref{formatting}.}
We make these demands as optional \verb|\documentclass| arguments:
\begin{leftfigure}
\begin{lstlisting}
\documentclass[11pt, twoside]{article}
\end{lstlisting}
\end{leftfigure}

Other commands take no arguments at all---\verb|\LaTeX|,
which prints the \LaTeX{} logo, is one example.
These commands consume any space that follows them.
For example,
\begin{leftfigure}
\begin{lstlisting}
\LaTeX is great, but it can be a bit odd sometimes.
\end{lstlisting}
\end{leftfigure}
will give you
\begin{leftfigure}
\lm \LaTeX is great, but it can be a bit odd sometimes.
\end{leftfigure}
You can fix this with an empty pair of braces following the command.
Of course, braces aren't needed if there is no space to preserve:
\begin{leftfigure}
\begin{lstlisting}
Let's learn \LaTeX! \LaTeX{} is a powerful tool,
but a few of its rules are a little weird.
\end{lstlisting}
\end{leftfigure}
gets us
\begin{leftfigure}
\lm Let's learn \LaTeX! \LaTeX{} is a powerful tool,
but a few of its rules are a little weird.
\end{leftfigure}

\section{Special characters and line breaks}

Some characters have special meanings in \LaTeX.
We saw above, for example, that \verb|%| starts a comment
and \verb|\| starts a command.
The full list of special characters is:
\begin{leftfigure}
\begin{lstlisting}
# $ % ^ & _ { } ~ \
\end{lstlisting}
\end{leftfigure}
Each has a corresponding command
for actually printing it in your document. Respectively, they are:
\begin{leftfigure}
\begin{lstlisting}
\# \$ \% \^{} \& \_ \{ \} \~{} \textbackslash
\end{lstlisting}
\end{leftfigure}
Regardless of whatever follows them,
the caret (\,\texttt{\^{}}\,) and tilde (\,\~{}\,) always need braces.
This is a relic from days when these commands produced
\introduce{diacritical marks}:
once upon a time, \LaTeX{} users would typeset ``jalapeño'' with
\verb|jalape\~no|.
Today we just type \texttt{ñ} into our source
file.\punckern\footnote{This can be easier or harder depending
on your keyboard, your editor, and your language settings in your \acronym{os}.
We'll talk more about languages and Unicode fun in \chapref{i18n}.}

If you're wondering why we print \texttt{\textbackslash} with
\verb|\textbackslash| instead of \verb|\\|\,,
it's because the latter is the command to force a line break.
\begin{leftfigure}
\begin{lstlisting}
Give me \\
a brand new line!
\end{lstlisting}
\end{leftfigure}
obeys:
\begin{leftfigure}
\lm Give me \\
a brand new line!
\end{leftfigure}
Use this power sparingly---deciding how to elegantly break paragraphs into
lines is one of \LaTeX{}'s greatest skills.

\section{Environments}

We often format text in \LaTeX{} by placing it in \introduce{environments}.
These always start with \verb|\begin{name}| and conclude with \verb|\end{name}|,
where \texttt{name} is that of the desired environment.
Take the \texttt{quote} environment,
which adds additional space on both sides of a block quotation:
\begin{leftfigure}
\begin{lstlisting}
Donald Knuth once wrote,
\begin{quote}
We should forget about small efficiencies,
say about 97\% of the time:
premature optimization is the root of all evil.
Yet we should not pass up our opportunities in
that critical 3\%.
\end{quote}
\end{lstlisting}
\end{leftfigure}
produces
\begin{leftfigure}
\lm
Donald Knuth once wrote,
\begin{quote}
We should forget about small efficiencies,
say about 97\% of the time:
premature optimization is the root of all evil.
Yet we should not pass up our opportunities in
that critical 3\%.
\end{quote}
\end{leftfigure}

\section{Groups and command scope}
Some commands change how \LaTeX{} sets the text that follows them.
\verb|\itshape|, for example, \textit{italicizes} everything that comes after it.
To limit a command's influence to a certain area, surround it with braces.
\begin{leftfigure}
\begin{lstlisting}
{\itshape Sometimes we want italics}, but only sometimes.
\end{lstlisting}
\end{leftfigure}
becomes
\begin{leftfigure}
\lm {\itshape Sometimes we want italics}, but only sometimes.
\end{leftfigure}
The braced region is called a \introduce{group},
and commands placed inside a group lose their power once it ends.
Environments also create their own groups:
\begin{leftfigure}
\begin{lstlisting}
\begin{quote}
\itshape If I italicize a quote, the following text will
use upright type again.
\end{quote}
See? Back to normal.
\end{lstlisting}
\end{leftfigure}
typesets
\begin{leftfigure}
\lm
\begin{quote}
\itshape If I italicize a quote, the following text will
use upright type again.
\end{quote}
See? Back to normal.
\end{leftfigure}
Groups can also be used to handle the spacing oddities of zero-argument
commands: some prefer \verb|{\LaTeX}| over \verb|\LaTeX{}|.

\chapter{Document Structure}
\label{structure}

Every \LaTeX{} document is different,
but all share a few common elements.

\section{The preamble and packages}
In the last chapter, you built your first document with:
\begin{leftfigure}
\begin{lstlisting}
\documentclass{article}

\begin{document}
Hello, World!
\end{document}
\end{lstlisting}
\end{leftfigure}
The space between \verb|\documentclass| and the start of the
\texttt{document} environment is called the \introduce{preamble}.
Here we handle whatever setup we need, including importing packages.
These add new commands, or modify the document in interesting ways.
The ones in your \LaTeX{} distribution come from the Comprehensive \TeX{}
Archive Network---or \acronym{ctan}---at \https{ctan.org}.\punckern\footnote{%
Curious readers might wonder what \TeX{} is, and how it differs from \LaTeX.
The short answer is that \TeX{} is the system that \LaTeX{}
is built on top of---the latter is a framework of commands for the former.
(\texttt{\textbackslash documentclass} and friends come from
\LaTeX{}, for example, but \TeX{} is what ultimately lays out a document.)
A longer answer is at the end of this guide under Appendix~\ref{history}.
We won't discuss how to use plain \TeX{} here. That's for another book---The
\TeX book.}
You'll also find packages' manuals there,
so make it your first stop when learning how to use one.

To import a package, add a \verb|\usepackage| command
with its name as the argument.
As a simple example, let's write a document with the \texttt{metalogo}
package, which adds \verb|\LuaLaTeX| and \verb|\XeLaTeX|:
\begin{leftfigure}
\begin{lstlisting}
\documentclass{article}

\usepackage{metalogo}

\begin{document}
\XeLaTeX{} and \LuaLaTeX{} are neat.
\end{document}
\end{lstlisting}
\end{leftfigure}
\begin{samepage}
should get you a \textsc{pdf} that reads
\begin{leftfigure}
\lm \XeLaTeX{} and \LuaLaTeX{} are neat.
\end{leftfigure}
\end{samepage}
\verb|\usepackage| accepts optional arguments
and passes them to whatever code you're importing.
The \texttt{geometry} package, for instance,
takes your desired paper size and margins.
For \acronym{us}~\textsc{l}etter paper with one-inch margins,
type:
\begin{leftfigure}
\begin{lstlisting}
\usepackage[
    letterpaper,
    left=1in, right=1in, top=1in, bottom=1in
]{geometry}
\end{lstlisting}
\end{leftfigure}
Arguments can be spaced however you like,
so long as there are no empty lines between them.

\section{Hierarchy}

Authors often split their writing into sections to help readers navigate it.
\LaTeX{} offers seven different commands to break up your documents:
\verb|\part|, \verb|\chapter|, \verb|\section|, \verb|\subsection|,
\verb|\subsubsection|, \verb|\paragraph|, and \verb|\subparagraph|.
Issue the command where you want an area to start,
providing its name as the argument.
For example,
\begin{leftfigure}
\begin{lstlisting}
\documentclass{book}

\begin{document}

\chapter{The Start}
This is a very short chapter in a very short book.

\chapter{The End}
Is the book over yet?

\section{No!}
There's some more we must do before we go.

\section{Yes!}
Goodbye!
\end{document}
\end{lstlisting}
\end{leftfigure}
Some levels are only available in certain document classes---chapters,
for example, only appear in books.
And don't go too crazy with these commands.
Most works just need a few levels to organize them.

These bits of structure are automatically numbered.
The title of this chapter was produced with \verb|\chapter{Document Structure}|,
and \LaTeX{} figured out that it was chapter~\ref{structure}.

\section{What next?}

As promised, this book isn't meant as a comprehensive reference,
but it \emph{will} point you to places where you can learn more.
We'll wrap up most chapters with some related topics that you can
explore on your own.

Consider learning how to:
\begin{itemize}
\item Automatically start your document with its title, your name,
    and the date using \verb|\maketitle|.
\item Build a table of contents
    with \verb|\tableofcontents|.
\item Control section numbering with \verb|\setcounter{secnumdepth}|
or starred commands like \verb|\subsection*{foo}|.
\item Create cross-references with \verb|\label| and \verb|\ref|.
\item Use KOMA~Script, a set of document classes and packages
to customize nearly every aspect of your document,
from heading fonts to footnotes.
\item Include images with the \texttt{graphicx} package.
\item Add hyperlinks with the \texttt{hyperref} package.
\item Split large documents into multiple files using \verb|\input|.
\end{itemize}

\chapter{Formatting Text}
\label{formatting}

\section{Emphasis}

Sometimes you need some extra punch to get your point across.
The simplest way to emphasize text in \LaTeX{} is with the \verb|\emph| command,
which \emph{italicizes} its argument:
\begin{leftfigure}
\begin{lstlisting}
\emph{Oh my!}
\end{lstlisting}
\end{leftfigure}
gives us
\begin{leftfigure}
\lm \emph{Oh my!}
\end{leftfigure}
We have more tools at our disposal:
\begin{leftfigure}
\begin{lstlisting}
We can also use \textbf{boldface} or \textsc{small caps}.
\end{lstlisting}
\end{leftfigure}
producing
\begin{leftfigure}
\lm%
We can also use \textbf{boldface} or \textsc{small caps}.
\end{leftfigure}
Be judicious when you use, especially boldface.
It excels at drawing the reader's attention away from everything around it,
so too much is distracting.

\section{Meeting the whole (type) family}

The styles shown above are just a few of the many available to you.
A (mostly) complete list follows:
\begin{flushleftfigure}
\lm%
\begin{tabularx}{0.9\textwidth}{l|l|l}
{\normalfont Command} & {\normalfont Alternative} & {\normalfont Style} \\
\hline
\texttt{\textbackslash textnormal\{...\}} & \texttt{\{\textbackslash normalfont ...\}} & the default \\
\texttt{\textbackslash emph\{...\}} & \texttt{\{\textbackslash em ...\}} & \emph{emphasis, typically italics} \\
\texttt{\textbackslash textrm\{...\}} & \texttt{\{\textbackslash rmfamily ...\}} & roman (serif) type \\
\texttt{\textbackslash textsf\{...\}} & \texttt{\{\textbackslash sffamily ...\}} & {\fontspec{Latin Modern Sans}sans serif type} \\
\texttt{\textbackslash texttt\{...\}} & \texttt{\{\textbackslash ttfamily ...\}} & {\fontspec{Latin Modern Mono}teletype (monospaced)} \\
\texttt{\textbackslash textit\{...\}} & \texttt{\{\textbackslash itshape ...\}} & \textit{italics} \\
% WTF: LuaTeX font loading doesn't seem to know what to do with Latin Modern Roman Slant
\texttt{\textbackslash textsl\{...\}} & \texttt{\{\textbackslash slshape ...\}} & {\fontspec{lmromanslant10-regular}slanted, or oblique type} \\
\texttt{\textbackslash textsc\{...\}} & \texttt{\{\textbackslash scshape ...\}} & \textsc{Small Capitals} \\
\texttt{\textbackslash textbf\{...\}} & \texttt{\{\textbackslash bfseries ...\}} & \textbf{boldface} \\
\end{tabularx}
\end{flushleftfigure}
Prefer the first form (which takes the text to format as an argument)
over the second
(which affect the group they are issued in),
since the former automatically handle any
spacing corrections needed around them.\punckern\footnote{For example,
\textit{italic type} amidst upright type needs to be followed
by a slight amount of additional space, called an ``italic correction''\quotekern.}
However, when formatting multiple paragraphs,
or when defining the style of other commands,\punckern\footnote{%
For instance, this book's section headers are styled with
\texttt{\textbackslash Large\allowbreak\textbackslash itshape}.}
the second variety is the only option.

\section{Sizes}

The font size of \introduce{body text}---that is, your main content---is
usually ten points,\punckern\footnote{The standard digital publishing point,
sometimes called the PostScript point, is \otffrac{1}{72} of an inch.
\LaTeX{}, for historical reasons, defines its point (\texttt{pt})
as \otffrac{100}{7227} of an inch
and the former as ``big points''\quotekern, or \texttt{bp}.
Use whichever you'd like.}
but can be adjusted by passing arguments to
\verb|\documentclass|.\punckern\footnote{Stock \LaTeX{} classes accept
\texttt{10pt}, \texttt{11pt}, or \texttt{12pt} as optional arguments.
KOMA~Script classes accept arbitrary sizes with
\monobox{fontsize=<size>}.}
To scale text relative to this default size, use the following commands:
\begin{flushleftfigure}
\lm%
\renewcommand{\arraystretch}{1.1}%
\begin{tabular}{l l}
\texttt{\textbackslash tiny} & \tiny Example Text \\
\texttt{\textbackslash scriptsize} & \scriptsize Example Text \\
\texttt{\textbackslash footnotesize} & \footnotesize Example Text \\
\texttt{\textbackslash small} & \small Example Text \\
\texttt{\textbackslash normalsize} & \normalsize Example Text \\
\texttt{\textbackslash large} & \large Example Text \\
\end{tabular}
\end{flushleftfigure}
\clearpage
\begin{flushleftfigure}
\lm%
\begin{tabular}{l l}
\texttt{\textbackslash Large} & \Large Example Text \\
\texttt{\textbackslash LARGE} & \LARGE Example Text \\
\texttt{\textbackslash huge} & \huge Example Text \\
\texttt{\textbackslash Huge} & \Huge Example Text \\
\end{tabular}
\end{flushleftfigure}
You'll find some subtleties at play here.
\LaTeX's default type family, Latin Modern,
comes in multiple \introduce{optical sizes}.
Smaller fonts aren't just shrunken versions of their big siblings---they
have thicker strokes, exaggerated features,
and more generous spacing to improve legibility at their size.
\begin{leftfigure}
\fontspec{lmroman5-regular} When I magnify 5 point type
\lm and place the result next to 11 point type,
you can instantly see the differences.
\end{leftfigure}
Optical sizes were standard back when fonts were made out of metal,
but many digital typefaces lack them,
given how much more work it demands from the type
designer.\punckern\footnote{If you are fortunate enough to have
a typeface with multiple optical sizes, \LuaLaTeX{}
and \XeLaTeX{} can make good use of them! See \chapref{fonts}
for more on font selection.}

But points and optical sizes don't tell the whole story.
Each typeface has its own proportions, which make a huge difference
in perceived size.
(Compare Garamond, {\fontspec{Latin Modern Roman} Latin Modern},
and {\fontspec{NHaasGroteskDSPro-45Lt}\addfontfeature{LetterSpace=3}Helvetica}, all at 11 points.)
Shown below are some common terms:
\begin{centerfigure}
\includegraphics[keepaspectratio,width=0.7\textwidth]{heights.png}

\captionof{figure}{Type sits on the \introduce{baseline},
rises to its \introduce{ascender height},
and drops to its \introduce{descender height}.
The \introduce{cap height} refers to the size of uppercase letters,
and the \introduce{x-height} refers to the size of lowercase letters.}
% For size reference:
%{\sffamily\fontsize{8pt}{8pt}\selectfont This is 8-point text.}
\end{centerfigure}

If the previous commands don't give you a size you need,
you can create custom ones with \verb|\fontsize|,
which takes both a text size and a
distance between baselines.
This must be followed with \verb|\selectfont| to take effect.
For example, \texttt{\textbackslash fontsize\{30pt\}\allowbreak\{30pt\}%
\allowbreak\textbackslash selectfont}
produces
\begin{leftfigure}
\lm
\fontsize{30pt}{30pt}\selectfont
large type with no \\
additional space \\
between lines
\end{leftfigure}
{\fontsize{11pt}{11pt}\selectfont
Note how without a bit of this extra space,
or \introduce{leading},\punckern\footnote{This term comes from the days of
metal type, when strips of lead or brass were inserted
between lines to space them out.\endnote{Jan Middendorp, \textit{Shaping Text}
(Amsterdam, 2014), 71}}
descenders from one line almost collide with ascenders and capitals on
the line below.
Leading is important---without it, blocks of text become uncomfortable to
read, especially at normal body sizes.\par}
Let your type breathe!\footnote{For a discussion of how much leading
to use, see \textit{Practical Typography},
as mentioned in Appendix~\ref{resources}.}

\section{What next?}
\begin{itemize}
\item Learn how to underline text with the \texttt{ulem}
    package.\punckern\footnote{Other typographical tools---like italics,
    boldface, and small caps---are generally preferable to underlining,
    but it has its uses.}
\item Use KOMA~Script to change the size and style of your section headings.
\item Learn the difference between italic and oblique type.
\item Change the default text style
    (used by \verb|\textnormal| and \verb|\normalfont|) by redefining
    \verb|\familydefault|.
\end{itemize}

\chapter{Punctuation}
\label{punctuation}

You would much rather encounter a panda that eats
shoots and leaves than one that eats, shoots,
and leaves.\punckern\endnote{Lynne Truss,
\textit{Eats, Shoots \& Leaves} (New York, 2003)}
Punctuation is a vital part of writing,
and there's more to it than your keyboard suggests.

\section{Quotation marks}

\LaTeX{} doesn't automatically convert \texttt{"}straight\texttt{"} quotes
into correctly-facing ``curly'' ones:
\begin{leftfigure}
\begin{lstlisting}
"This isn't right."
\end{lstlisting}
\end{leftfigure}
will get you
\begin{leftfigure}
\lm%
"This isn't right."
\end{leftfigure}
Instead, use \texttt{\bfseries `} for opening quotes and \texttt{\bfseries '} for closing
quotes.\punckern\footnote{If your keyboard happens to have keys for
``curly'' quotes (\,\texttt{“}\,\texttt{”}\,), feel free to use those instead!
Also, don't use \texttt{"} for closing double quotes.
Not only does \texttt{``example"} look a bit unbalanced,
but \texttt{"} is used as a formatting command when typesetting certain
languages, like German. (See \chapref{i18n} for more on international
typesetting.)}
\begin{leftfigure}
\begin{lstlisting}
``It depends on what the meaning of the word `is' is.''
\end{lstlisting}
\end{leftfigure}
quotes a former \acronym{us} president as,
\begin{leftfigure}
\lm%
``It depends on what the meaning of the word `is' is.''
\end{leftfigure}

\section{Hyphens and dashes}

Though they look similar,
hyphens (\,-\,), en dashes (\,--\,),
em dashes (\,---\,), and minus signs (\,$-$\,)
serve different purposes.
\begin{description}
\item[Hyphens] have a few applications:\endnote{Matthew Butterick,
    ``Hyphens and dashes''\quotekern,
    \textit{Practical Typography},
    \https{practicaltypography.com/hyphens-and-dashes.html}}
    \begin{itemize}[leftmargin=*]
    \item They allow a word to be split between the end of one line and the
        start of the next.
        \LaTeX{} usually handles this automatically.
    \item Some compound words use hyphens, like \emph{long-range}
        and \emph{field-effect}.
    \item They are used in phrasal adjectives.
        If I ask for ``five dollar bills''\punckern,
        do I want five \$1 bills, or several \$5 bills?
        It's clearer that I want the latter when typeset as
        \emph{five-dollar bills}.
    \end{itemize}
    Unsurprisingly, they are produced with the hyphen character (\,\texttt{-}\,).

\item[En dashes] are for ranges such as ``pages 4--12''\quotekern,
    and connected words, like ``the \mbox{US--Canada} border''\quotekern.
    \LaTeX{} places one wherever it sees two adjacent hyphens
    (\,\texttt{--}\,).

\item[Em dashes] are used to separate clauses of a sentence.
    Other punctuation---like parenthesis and commas---play a similar
    role.
    Em dashes are typeset with three hyphens (\,\texttt{---}\,).

\item[Minus signs] are used for negative quantities and
    mathematical expressions.
    They are similar in length to an en dash,
    but sit at a different height.
    Minus signs are set with \verb|\textminus|, or with the hyphen character
    when in a math environment (see \chapref{math}).
\end{description}

\section{Ellipses}

A set of three dots used to indicate a pause or omission is called an
\introduce{ellipsis}.
It is set with \verb|\dots|.
\begin{leftfigure}
\begin{lstlisting}
I'm\dots{} not sure.
\end{lstlisting}
\end{leftfigure}
becomes
\begin{leftfigure}
\lm%
I'm\dots{} not sure.
\end{leftfigure}
Ellipses are spaced differently than consecutive periods.
Don't use the latter as a poor substitute for the former.

\section{Spacing}

As we discovered in \chapref{hello},
\LaTeX{} inserts additional space between periods and whatever
follows them---presumably the start of the next sentence.
This isn't always what we want!
Consider honorifics like Mr.\ and Ms., for example.
In situations like these, we also need to prevent \LaTeX{} from starting a
new line after the period.
This calls for a \introduce{non-breaking space}, which we set with a tilde.
\begin{leftfigure}
\begin{lstlisting}
Please call Ms.~Shrdlu.
\end{lstlisting}
\end{leftfigure}
produces proper spacing:
\begin{leftfigure}
\lm%
Please call Ms.~Shrdlu.
\end{leftfigure}

In other occasions, such as when we abbreviate units of
measurement,\punckern\footnote{There are also dedicated packages for doing so,
like \texttt{siunitx}.}
we want thinner spaces than our usual inter-word ones.
For these, we use \verb|\,|\,:
\begin{leftfigure}
\begin{lstlisting}
Launch in 2\,h 10\,m.
\end{lstlisting}
\end{leftfigure}
announces
\begin{leftfigure}
\lm%
Launch in 2\,h 10\,m.
\end{leftfigure}

\section{What next?}
\begin{itemize}
\item Learn other commands for spacing, such as \verb|\:|, \verb|\;|,
    \verb|\enspace|, and \verb|\quad|.
\item Use the \texttt{csquotes} package's \verb|\enquote| to simplify
    nested quotations, e.g., \\
    \enquote{She exclaimed, \enquote{I can't believe it!}}
\item Discover the typographical origins of terms like \introduce{en},
    \introduce{em}, and \introduce{quad}.
\item Familiarize yourself with the difference between \texttt{/} and
    \verb|\slash|.
\item Add hyphenations for uncommon words using \verb|\hyphenate|
    or \verb|\-|\,.\punckern\footnote{\LaTeX{} usually does a good
    job of automatically hyphenating words, based on a dictionary of patterns
    stored for each language. You should rarely need to use these commands.}
\end{itemize}

\chapter{Layout}

\section{Justification and alignment}

\LaTeX{} justifies text remarkably well.
Instead of considering lines individually---as most word processors,
web browsers, and e-readers do---it examines all possible line breaks
in a given paragraph,
then picks whichever ones give the best overall
spacing.\punckern\endnote{Donald E.~Knuth and Michael F.~Plass,
\textit{Breaking Paragraphs Into Lines} (Stanford, 1981)}
Combined with its ability to automatically hyphenate words,
which permits line breaks in many more places,\punckern\endnote{%
Franklin Mark Liang,
\textit{Word Hy-phen-a-tion by Com-put-er} (Stanford, 1983),
\http{www.tug.org/docs/liang/}}
this approach produces better paragraph layouts than almost any other software.

But sometimes we don't want our text justified.
If you would like it to be flush left,
place it in a \texttt{flushleft} environment
or add \verb|\raggedright| to the current group.
To center it, place it in a
\texttt{center} environment or add \verb|\centering| to the current group.
And to flush it against the right margin,
use a \texttt{flushright} environment or \verb|\raggedleft|.

\begin{leftfigure}
\begin{lstlisting}
\begin{flushleft}
This text is flush left, with a ragged right edge.
Some prefer this layout since the space between words
is more consistent than it is in justified text.
\end{flushleft}

\begin{center}
This text is centered.
\end{center}

\begin{flushright}
And this text is flush right.
\end{flushright}
\end{lstlisting}
\end{leftfigure}
sets
\begin{flushleftfigure}
\begin{minipage}{0.99\textwidth}
\lm%
\begin{flushleft}
This text is flush left, with a ragged right edge.
Some prefer this layout since the space between words
is more consistent than it is in justified text.
\end{flushleft}

\begin{center}
This text is centered.
\end{center}

\begin{flushright}
And this text is flush right.
\end{flushright}
\end{minipage}
\end{flushleftfigure}

\section{Lists}

\LaTeX{} provides multiple environments for creating lists:
\texttt{itemize}, \texttt{enumerate}, and \texttt{description}.
In all three, each item starts with an \verb|\item| command.
The first environment, \texttt{itemize}, creates bulleted lists.
With:
\begin{leftfigure}
\begin{lstlisting}
\begin{itemize}
\item 5.56 millimeter
\item 9 millimeter
\item 7.62 millimeter
\end{itemize}
\end{lstlisting}
\end{leftfigure}
you get
\begin{leftfigure}
\lm%
\begin{itemize}[leftmargin=*]
\item 5.56 millimeter
\item 9 millimeter
\item 7.62 millimeter
\end{itemize}
\end{leftfigure}

\bigskip
\noindent \texttt{enumerate} numbers its lists:
\begin{leftfigure}
\begin{lstlisting}
\begin{enumerate}
\item Collect underpants
\item ?
\item Profit
\end{enumerate}
\end{lstlisting}
\end{leftfigure}
produces
\begin{leftfigure}
\lm%
\begin{enumerate}[leftmargin=*]
\item Collect underpants
\item ?
\item Profit
\end{enumerate}
\end{leftfigure}

\bigskip
\noindent The \texttt{description} environment starts each item with some
emphasized \introduce{label},
then indents all subsequent lines for that item:
\begin{leftfigure}
\begin{lstlisting}
\begin{description}
\item[Alan Turing] was a British mathematician who
    laid much of the groundwork for computer science.
    He is perhaps most remembered for his model of
    computation, the Turing machine.
\item[Edsger Dijkstra] was a Dutch computer scientist.
    His contributions in many subdomains---such as
    concurrency and graph theory---are still in wide use.
\item[Leslie Lamport] is an American computer scientist.
    He defined the concept of sequential consistency,
    which is used to safely communicate between tasks
    running in parallel.
\end{description}
\end{lstlisting}
\end{leftfigure}
gives us
\begin{leftfigure}
\lm%
\begin{description}[leftmargin=*]
\item[\lm Alan Turing] was a British mathematician who
    laid much of the groundwork for computer science.
    He is perhaps most remembered for his model of computation,
    the Turing machine.
\item[\lm Edsger Dijkstra] was a Dutch computer scientist.
    His contributions in many subdomains---such as
    concurrency and graph theory---are still in wide use.
\item[\lm Leslie Lamport] is an American computer scientist.
    He defined the concept of sequential consistency,
    which is used to safely communicate between tasks
    running in parallel.
\end{description}
\end{leftfigure}

\section{Columns}

We often split layouts into several columns, especially when printing on
\textsc{a4} or \acronym{us}~\textsc{l}etter paper,
since it allows more comfortable line widths at standard
8--12\,pt text sizes.\punckern\footnote{You'll see different advice depending
on where you look, but as a rule of thumb, design layouts to have
between 45 and 80 characters (including spaces) per line.
If a line is too long, readers have an uncomfortable time scanning for
the start of the next one.
If a line is too short, it doesn't have much inter-word spacing to adjust,
which can lead to odd gaps or excessive hyphenation.}
You can either add the \texttt{twocolumn} option to your document class,
which splits everything in two, or you can use the \texttt{multicols}
environment from the \texttt{multicol} package:
\begin{leftfigure}
\begin{lstlisting}
One nice feature of \texttt{multicol} is that you can
combine arbitrary layouts.
\begin{multicols}{2}
This example starts with one column,
then sets the following section as two.
The \texttt{multicols} environment splits the text
inside it so that each column is about the same height.
\end{multicols}
\end{lstlisting}
\end{leftfigure}
is split into
\begin{leftfigure}
\lm%
One nice feature of {\lt multicol} is that you can
combine arbitrary layouts.
\begin{multicols}{2}
This example starts with one column,
then sets the following section as two.
The \mbox{\lt multicols} environment splits the text
inside it so that each column is about the same height.
\end{multicols}
\end{leftfigure}

\section{Page breaks}

Some commands, like a book's \verb|\chapter|, insert page breaks.
You can add your own with \verb|\clearpage|.
If you are using the \texttt{twoside} document class option for double-sided
printing, you can break to the front of the next page with
\verb|\cleardoublepage|.

\section{Footnotes}

Footnotes are useful for inserting references,
or for remarks that readers might find helpful,
but aren't crucial to the main text.
The \verb|\footnote| command places a marker at its location in the
body text, then sets its argument at the bottom of the current
page:
\begin{leftfigure}
\begin{lstlisting}
I love footnotes!\footnote{Perhaps a bit too much\dots}
\end{lstlisting}
\end{leftfigure}
proclaims
\begin{leftfigure}
\lm%
I love footnotes!\footnote{Perhaps a bit too much\dots}
\end{leftfigure}

\section{What next?}
\begin{itemize}
\item Control paragraph spacing, either with
KOMA~Script options, or with the \texttt{parskip} package.
\item Set the page size and margins with the \texttt{geometry} package.
\item Customize list formatting with the \texttt{enumitem} package.
\item Create tables with the \texttt{tabular} environment.
\item Align text with tab stops using the \texttt{tabbing} environment.
\item Customize footnote symbols and layout with the
    \texttt{footmisc} package and KOMA~Script.
\item Create horizontal and vertical space with commands such as
    \verb|\vspace|, \verb|\hspace|, \verb|\vfill|, and \verb|\hfill|.
\item Learn what units \LaTeX{} provides for specifying spacing. \\
    (We've already mentioned a few here, like
    \texttt{pt}, \texttt{bp}, and \texttt{in}.)
\end{itemize}

\chapter{Mathematics}
\label{math}

\LaTeX{} excels at typesetting mathematics, both inside body text
($x_n^2+y_n^2=r^2$) and on their own lines:
\[\sum_{n=0}^{\infty} \frac{f^{(n)} (a)}{n!} (x - a)^n\]
The former is typed inside \verb|$...$| or \verb|\(...\)|,
and the latter within \verb|\[...\]|.
In these math environments, the rules of \LaTeX{} change:
\begin{itemize}
\item Most spaces and line breaks are ignored.
    Spacing decisions are made for you based on
    typographical conventions for mathematics.
    \verb|$x+y+z$| and \verb|$x + y + z$| both give you $x+y+z$.
\item Empty lines are not allowed---each formula occupies a single
    ``paragraph''\quotekern.
\item Letters are automatically italicized, as they are assumed to be variables.
\end{itemize}
To return to normal ``text mode'' inside a formula, use the \verb|\text| command.
Standard formatting commands work in these blocks.
From
\begin{leftfigure}
\begin{lstlisting}
\[ \text{fake formulas} = \textbf{annoyed mathematicians} \]
\end{lstlisting}
\end{leftfigure}
we get
\[ \text{\lm fake formulas} = \textbf{\lm annoyed mathematicians} \]

\section{Examples}

Typesetting mathematics is arguably the raison d'être of
\LaTeX,\punckern\footnote{Well, \TeX} but we could
take dozens of pages to just cover the basics.
Given the breadth of modern mathematics,
there are \emph{many} different commands and environments.
You owe it to yourself to find some real references and learn what \LaTeX{}
is capable of.
Before moving on, though, let's see some examples of what it can do.
\newpage

\begin{enumerate}
\item \verb|x = \frac{-b \pm \sqrt{b^2 - 4ac}}{2a}|
    \[x = \frac{-b \pm \sqrt{b^2 - 4 a c}}{2a} \]

\item \verb|e^{j \theta} = \cos(\theta) + j \sin(\theta)|
    \[e^{j \theta} = \cos(\theta) + j \sin(\theta)\]

\item
\begin{verbatim}
\begin{bmatrix}
x' \\
y'
\end{bmatrix} =
\begin{bmatrix}
\cos \theta &  -\sin\theta \\
\sin \theta & \cos \theta
\end{bmatrix}
\begin{bmatrix}
x \\
y
\end{bmatrix}
\end{verbatim}
\[
\begin{bmatrix}
x' \\
y'
\end{bmatrix} =
\begin{bmatrix}
\cos \theta &  -\sin\theta \\
\sin \theta & \cos \theta
\end{bmatrix}
\begin{bmatrix}
x \\
y
\end{bmatrix}
\]

\item
\begin{verbatim}
\oint_{\partial \Sigma} \mathbf{E} \cdot
\mathrm{d}\boldsymbol{\ell}
= - \frac{\mathrm{d}}{\mathrm{d}t}
    \iint_{\Sigma} \mathbf{B} \cdot \mathrm{d}\mathbf{S}
\end{verbatim}
    \[\oint_{\partial \Sigma} \mathbf{E} \cdot \mathrm{d}\boldsymbol{\ell}  = - \frac{\mathrm{d}}{\mathrm{d}t} \iint_{\Sigma} \mathbf{B} \cdot \mathrm{d}\mathbf{S}\]
\end{enumerate}

\chapter{Fonts}
\label{fonts}

Digital fonts have completely changed since \LaTeX{} was created decades ago.
\LaTeX{} originally used \MF,
a format Donald Knuth designed specifically for \TeX{}.
As time went on, support for PostScript\footnote{One of
Adobe's original claims to fame,
PostScript is a language for defining and drawing computer graphics,
including type. It remains in widespread use today.} fonts was added.
Today, \LuaLaTeX{} and \XeLaTeX{} support the modern font formats you'll find
on your computer:
TrueType and OpenType.\punckern\footnote{Mac versions of \LaTeX{} also support
Apple's \acronym{aat}, but let's limit this discussion to
more ubiquitous formats.}

\begin{description}
\item[TrueType] was developed by Apple and Microsoft in the late 1980s.
    Most fonts that come pre-installed on your system are probably
    in this format.
    TrueType files generally end in a \monobox{.ttf} extension.
\item[OpenType] was first released by Microsoft and Adobe in 1996.
    Improvements over TrueType include its ability to embed
    various features, such as alternative glyphs
    and spacing options, into a single file.
    OpenType files usually end in an \monobox{.otf} extension.
\end{description}

\section{Changing fonts}

By default, \LuaLaTeX{} and \XeLaTeX{} use Latin Modern,
an OpenType rendition of \LaTeX's original type family, Computer Modern.
While these are high-quality fonts,
they're probably not the only ones you ever want to use.
For others, we turn to the \texttt{fontspec} package:
\begin{leftfigure}
\begin{lstlisting}
\documentclass{article}

\usepackage{fontspec}
\setmainfont[Ligatures=TeX]{Source Serif Pro}
\setsansfont[Ligatures=TeX]{Source Sans Pro}
\setmonofont{Source Code Pro}

\begin{document}
Hello, Source type family! Neat---no? \\
\sffamily Let's try sans serif! \\
\ttfamily Let's try monospaced!
\end{document}
\end{lstlisting}
\end{leftfigure}
should produce something like\footnote{Assuming, of course,
that you have Adobe's open-source fonts installed.\punckern\endnote{Adobe's
open-source typefaces are freely available at
\https{github.com/adobe-fonts}}}
\begin{leftfigure}
\fontspec[Ligatures=TeX]{Source Serif Pro} Hello, Source type family! Neat---no? \\
\fontspec[Ligatures=TeX]{Source Sans Pro} Let's try sans serif! \\
\fontspec{Source Code Pro} Let's try monospaced!
\end{leftfigure}
The \verb|Ligatures=TeX| option lets you use the punctuation
shortcuts from \chapref{punctuation} (\texttt{--} for en dashes,
\texttt{``} and \texttt{''} for curly quotes, etc.)
instead of forcing you to enter the the corresponding characters,
which probably aren't on your keyboard.
You usually don't want these substitutions with monospaced type, though.
Text that uses it---such as code---is often meant to be printed
verbatim. \verb|"Hello!"| shouldn't turn into
\verb|“Hello!“|.

\section{Selecting font files}

\texttt{fontspec} usually does a good job of finding
the files you need for a given typeface,
especially if you just want the basic set of
upright,
\textit{italic},
\textbf{bold}, and
\textit{\textbf{bold italic}} fonts.
But typefaces can have many more than that.
The version of Futura used in this book, for example, comes in
{\fontspec[Scale=MatchLowercase]{Futura-Lig}light},
{\fontspec[Scale=MatchLowercase]{Futura-Boo}book},
{\fontspec[Scale=MatchLowercase]{Futura-Med}medium},
{\fontspec[Scale=MatchLowercase]{Futura-Dem}demi},
{\fontspec[Scale=MatchLowercase]{Futura-Bol}bold}, and
{\fontspec[Scale=MatchLowercase]{Futura-ExtBol}extra bold} weights.
Each of these weights has an upright and an oblique font.
A typeface could have other variations,
like {\fontspec[Scale=MatchLowercase]{FuturaSc-Boo}small capitals}\footnote{%
OpenType allows some styles, like small caps, to be placed in the same file(s)
as the ``main'' glyphs for a given weight.
If your font supports this, \texttt{fontspec}
will automatically switch to them whenever you use
\monobox{\textbackslash textsc} or \monobox{\textbackslash scshape}.
But for TrueType fonts,
and for OpenType fonts that don't leverage this feature,
you'll have to specify separate files.}
or multiple optical sizes (see chapter \ref{formatting}).

We might want to hand-pick weights to achieve a certain look,
or to better match the
other fonts in our document.\punckern\footnote{Compare how
{\fontspec[Scale=MatchLowercase]{Futura-Lig}the light,}
{\fontspec[Scale=MatchLowercase]{Futura-Boo}book,}
{\fontspec[Scale=MatchLowercase]{Futura-Med}and medium weights}
of Futura look next to the rest of the type on this page.}
Continuing to use Futura as an example,
say we want ``book'' as our default weight
and ``demi'' for bold.
Assuming the font files are named:
\begin{itemize}
\item \monobox{Futura-Boo} for
    {\fontspec[Scale=MatchLowercase]{Futura-Boo}upright book weight}
\item \monobox{Futura-BooObl} for
    {\fontspec[Scale=MatchLowercase]{Futura-BooObl}oblique book weight}
\item \monobox{FuturaSC-Boo} for
    {\fontspec[Scale=MatchLowercase]{FuturaSC-Boo}small caps, book weight}
\item \monobox{Futura-Dem} for
    {\fontspec[Scale=MatchLowercase]{Futura-Dem}upright demi(bold)}
\item \monobox{Futura-DemObl} for
    {\fontspec[Scale=MatchLowercase]{Futura-DemObl}oblique demibold}
\end{itemize}

\noindent Our setup might resemble:
\begin{leftfigure}
\begin{lstlisting}
\usepackage{fontspec}
\setmainfont[
    Ligatures=TeX,
    UprightFont = *-Boo,
    ItalicFont = *-BooObl,
    SmallCapsFont = *SC-Boo,
    BoldFont = *-Dem,
    BoldItalicFont = *-DemObl
]{Futura}
\end{lstlisting}
\end{leftfigure}
Note that instead of typing out \monobox{Futura-Boo},
\monobox{Futura-BooObl}, and so on, we can use \texttt{*} to insert the base
name.\punckern\footnote{This is
a place where \XeLaTeX{} and \LuaLaTeX{}
differ in a way that's noticeable to the casual user.
The former uses system libraries---such as FontConfig on Linux---to
find font files.
The latter has its own font loader,
based on code from FontForge.\punckern\endnote{\textit{\LuaTeX{} Reference}
(Version 1.0.4, February 2017), 10}
Because the two look for files in different ways,
the expected name of a font might differ between the two engines.
See the \texttt{fontspec} manual for details.}

\section{Scaling}

Creating a cohesive design with multiple fonts is tricky,
especially since they might look completely different
at the same point size.
\texttt{fontspec} can help by scaling fonts to match either the
x-height or the cap height of your main font with
\verb|Scale=MatchLowercase| or \verb|Scale=MatchUppercase|,
respectively.
But one way to sidestep this issue is to use fewer
typefaces in the first place.
Just one or two, used carefully, can produce amazing results.


\section{OpenType features}

As mentioned at the start of the chapter,
OpenType fonts provide various features that can be switched on and off.
In \LaTeX{}, we do this with optional arguments to
\verb|\setmainfont| and friends.
Features can also be set for the current group with
\verb|\addfontfeature|.
Let's examine some common ones.

\subsection{Ligatures}

Many typefaces use \introduce{ligatures}, which combine multiple characters
into a single glyph.\punckern\footnote{Ligatures fell out
of style during the 20{\addfontfeature{VerticalPosition=Superior}th}
century due to limitations of printing technology and the increased popularity
of sans serif typefaces, which often lack them.
Today they are making a comeback,
thanks in no small part to their support in OpenType.}
OpenType groups them into three categories:
\begin{description}
\item[Standard] ligatures remedy spacing problems between certain characters.
    Consider lowercase f and i: in many typefaces,
    these combine to form the ligature fi to avoid
    awkward spacing between f's ascender and i's dot
    {\addfontfeature{Ligatures=CommonOff} (\,fi\,)}.
    Other common examples in English writing include ff,
    ffi, fl, and ffl.
    Standard ligatures are enabled by default.
\item[Discretionary] ligatures, such as
    {\addfontfeature{Ligatures=Discretionary}ct},
    are offered by some fonts.
    They're disabled by default but enabled with
    \verb|Ligatures=Discretionary|.
\item[Historical] ligatures are ones which have fallen out of common use,
    such as those with a \introduce{long~s} (e.g., ſt).
    These are also disabled by default
    but can be enabled with \verb|Ligatures=Historic|.
\end{description}
Multiple options can be grouped together.
Say you want discretionary ligatures.
In the likely event that you also want \verb|Ligatures=TeX|,
you would enable both with
\verb|Ligatures={TeX,Discretionary}|.
Ligatures can also be disabled with corresponding \verb|*Off|
options. If you want to stop using discretionary ligatures for some passage,
\begin{leftfigure}
\begin{lstlisting}
{\addfontfeature{Ligatures=DiscretionaryOff}...}
\end{lstlisting}
\end{leftfigure}
does the trick.

Some words arguably look better without ligatures---\mbox{shelfful}
is a classic example.\punckern\endnote{Knuth, \textit{The \TeX book},
(Addison-Wesley, 1986), 19}
You can manually prevent a ligature by inserting a zero-width space, e.g.,
\verb|shelf\hspace{0pt}ful|.
Or, since life is too short, you can let the \texttt{selnolig} package
do this for you.

\subsection{Figures}

When setting figures,\punckern\footnote{%
We use the term \introduce{figures} in typographical contexts for the symbols
we might otherwise call \introduce{digits} (0, 1, 2, 3, 4, 5, 6, 7, 8, 9).}
you have two
choices to make: lining versus text,
and proportional versus tabular.
\introduce{Lining} figures, sometimes called \introduce{titling} figures,
have heights similar to capital letters:
\begin{leftfigure}
\addfontfeature{Numbers=LowercaseOff}
A B C D 1 2 3 4
\end{leftfigure}
\introduce{Text}, or \introduce{oldstyle} figures,
share more similarities with lowercase letters:
\begin{leftfigure}
Sitting cross-legged on the floor\dots{} 25 or 6 to 4?
\end{leftfigure}
Either choice is fine for body text, but don't mix capital letters with
text figures.
``F-15C'' looks odd, as does ``V2.3 Release''\quotekern.

{\addfontfeature{Numbers=LowercaseOff}
The terms \introduce{proportional} and \introduce{tabular} refer to spacing.
Tabular figures are set with a uniform width, so that 1 takes up
the same space as 8.
As their name suggests, this is great for tables and other scenarios
where figures line up in columns:}
\begin{leftfigure}
\addfontfeature{Numbers={Tabular,LowercaseOff}}
\begin{tabular}{l|c r}
Item & Qty. & Price \\
\hline
Gadgets & 42 & \$5.37 \\
Widgets & 18 & \$12.76 \\
\end{tabular}
\end{leftfigure}
Proportional figures are the opposite---their spacing is, well\dots{}
\emph{proportional} to the width of each figure.
They look a bit nicer in body text: 1837
looks more natural here than
{\addfontfeature{Numbers=Tabular}1837} does.

You select figures with the following options:
\begin{leftfigure}
\begin{tabular}{l l}
\texttt{Numbers=} & \texttt{Lining / Uppercase} \\
 & \texttt{OldStyle / Lowercase} \\
 & \texttt{Proportional} \\
 & \texttt{Tabular / Monospaced}
\end{tabular}
\end{leftfigure}
Like ligature options, these can be combined:
proportional lining figures are set
with \texttt{Numbers=\allowbreak\{Proportional,\allowbreak Lining\}},
and tabular oldstyle figures are set with
\texttt{Numbers=\allowbreak\{Tabular,\allowbreak OldStyle\}}.
Each option also has a corresponding \verb|*Off|
variant.\punckern\footnote{This is especially useful since different fonts
have different defaults.
Some fonts use lining figures by default and enable text figures
with \monobox{Numbers=OldStyle}.
Others default to text figures and require \monobox{Numbers=Lining}.}

Finally, some fonts provide \introduce{superior and inferior} figures,
which are used to set ordinals
(\otford{1}{st}, \otford{2}{nd} \otford{3}{rd}, \dots),
fractions (\,\otffrac{25}{624}\,), and so on.
They have the same weight as the rest of the font's characters,
offering a more consistent look than shrunken versions of full-sized figures.
(Compare the examples above to their imposters:
{\fontspec[OpticalSize=0]{garamondpremrpro}%
\mbox{1\textsuperscript{st}},
\mbox{2\textsuperscript{nd}},
\mbox{3\textsuperscript{rd}},
and
\,\mbox{\textsuperscript{25}^^^^2044\textsubscript{624}}%
\,}.
Notice how this second set is too light compared to the surrounding
type.)
Superior figures are typeset with
\texttt{VerticalPosition=\allowbreak Superior},
and inferiors are set with \texttt{VerticalPosition=\allowbreak Inferior}.

\section{What next?}
\begin{itemize}
\item Learn how \texttt{fontspec} can choose optical sizes based on
    point size, either automatically from ranges embedded in OpenType fonts,
    or manually using \texttt{SizeFeatures}.
\item Experiment with letter spacing---or \introduce{tracking}---with
    the \texttt{LetterSpace} option.
    Extra tracking is unnecessary in most cases,
    but can be useful to make \textsc{small caps}
    a little more \acronym{readable}.
\end{itemize}

\chapter{Microtypography}
\label{microtype}

\introduce{Microtypography} is the craft of improving text's legibility
with small, subliminal tweaks.
In other words, it is
\begin{quote}
\small
[\dots]the art of enhancing the appearance and readability of a
document while exhibiting a minimum degree of visual obtrusion.
It is concerned with what happens between or at the margins of characters,
words or lines. Whereas the macro-typographical aspects of a document
(i.e., its layout) are clearly visible even to the untrained eye,
micro-typographical refinements should ideally not even be recognisable.
That is, you may think that a document looks beautiful, but you
might not be able to tell exactly why: good micro-typographic practice tries to
reduce all potential irritations that might disturb a reader.\punckern\endnote{%
R Schlicht,
\textit{The microtype package}
(v2.7a, January 14, 2018), 4}
\end{quote}

In \LaTeX{}, microtypography is controlled with the
\texttt{microtype} package.
Its use is automatic---for the vast majority of documents, you should add
\begin{leftfigure}
\begin{lstlisting}
\usepackage{microtype}
\end{lstlisting}
\end{leftfigure}
to your preamble and move on.
But let's take a quick look at what the package actually does.

\section{Character protrusion}

By default, \LaTeX{} justifies lines between perfectly straight margins.
This is an obvious default,
but falls victim to an annoying optical illusion:
lines ending in small glyphs---like periods, commas,
or hyphens---seem shorter than lines that
don't.\punckern\footnote{Many other optical illusions come up in typography.
For example, if a circle, a square, and a triangle
of equal heights are placed next to each other,
the circle and triangle look smaller than the square.
For this reason, round or pointed characters (like O and A) must
be slightly taller than ``flat'' ones (such as H and T) for all
to appear the same height.\punckern\endnote{%
Jost Hochuli, \textit{Detail in typography}
(Éditions~\textsc{b}42, 2015),
18--19}}
\texttt{microtype} compensates by \introduce{protruding} these glyphs
into the margins.

\section{Font expansion}

To give paragraphs more even spacing and fewer hyphenated lines,
\texttt{microtype} can stretch characters horizontally.
You might think that distorting type like this would be immediately
noticeable,
but you're reading a book that does so on every page!
This effect, called \introduce{font expansion},
is applied \emph{very} slightly---by default,
character widths are altered by no more than two percent.\punckern\footnote{%
Of course, you can use package options to change this limit,
or disable the feature entirely.}

This feature isn't currently available for \XeLaTeX{}.
If you want to use it, you'll need \LuaLaTeX{}.

\section{What next?}

As always, see the package manual for ways to tweak these features.
\texttt{microtype} has a few other tricks,
but several only work on older \LaTeX{} engines.\punckern\footnote{i.e., pdf\TeX}
Those we care about---such as letterspacing---can be handled with
\texttt{fontspec} or other packages.

\chapter{Typographie Internationale}
\label{i18n}

Surprisingly, languages besides English exist.
You may want to write with them.

\section{Unicode}

Digitizing written language is a complicated topic that has evolved significantly
since \LaTeX's inception.
Today, most computer systems use Unicode to represent text. Briefly,
\begin{itemize}
\item A Unicode text file is a series of \introduce{code points}.
    Each represents a character to be drawn,
    an accent or diacritical mark to combine with an adjacent
    character,
    or formatting information,
    such as an instruction to print subsequent text right-to-left.
\item One or more of these code points combines to represent a
    \introduce{grapheme cluster} or \introduce{glyph},
    the shapes within fonts that we informally call ``characters''\quotekern.
\begin{centerfigure}
\large%
\fontspec[Ligatures=TeX]{NotoSerif}%
Приве́т
\quad\fontspec[Ligatures=TeX]{NotoSerif-Devanagari}%
नमस्ते
\captionof{figure}{How many characters do you see?
How many code points?}
\end{centerfigure}
\item Modern font formats contain encoding tables
    which map code points to the glyphs the file contains.
\end{itemize}
\LuaLaTeX{} and \XeLaTeX{} are Unicode-literate and play well with
Unicode text
files.\punckern\footnote{\LuaLaTeX{} accepts \mbox{\acronym{utf}-8} files.
\XeLaTeX{} also accepts \mbox{\acronym{utf}-16} and \mbox{\acronym{utf}-32}.}
Make sure that the fonts you select contain the glyphs you need---many
only support Latin languages.

\section{The polyglossia package}

When your document contains languages besides English,
consider using the \texttt{polyglossia} package.
It will automatically:
\begin{itemize}
\item Load language-specific hyphenations and other conventions.
\item Switch between user-specified fonts for each language.
\item Translate document labels,
    like ``chapter''\quotekern, ``section''\quotekern, and so on.
\item Format dates according to language-specific conventions.
\item Format numbers in languages that have their own numbering system.
\item Use the \texttt{bidi} package for documents with languages written
    right to left.
\item Set the script and language tags of OpenType fonts that have them.
\end{itemize}
To use \texttt{polyglossia}, specify your document's main language,
along with any others it uses.
Some languages also take regional dialects as an optional argument:
\begin{leftfigure}
\begin{lstlisting}
\usepackage{polyglossia}
\setdefaultlanguage[variant=american]{english}
\setotherlanguage{french}
\end{lstlisting}
\end{leftfigure}
Once set up,
\texttt{polyglossia} defines environments for the requested languages.
Each automatically applies their language's conventions to the text within.
French, for example, places extra space around punctuation, so
\begin{leftfigure}
\begin{lstlisting}
Dexter cried,
\begin{french}
«Omelette du fromage!»
\end{french}
\end{lstlisting}
\end{leftfigure}
gives
\begin{leftfigure}
\lm%
Dexter cried,
\begin{french}
\lm%
«Omelette du\footnote{Yes, it's \emph{omelette au fromage}.
Direct all complaints to Cartoon Network.} fromage!»
\end{french}
\end{leftfigure}

% FFS: https://github.com/reutenauer/polyglossia/issues/68
\directlua{polyglossia.desactivate_frpt()}

\section{What next?}
\begin{flushleft}
\begin{itemize}
\item See the \texttt{polyglossia} manual for language-specific commands.
\item Look into the \texttt{babel} package as an alternative to
    \texttt{polyglossia}.\punckern\footnote{\texttt{polyglossia} has better
    support for OpenType font features via \texttt{fontspec}.
    However, it is newer and has a few known bugs.
    \texttt{babel} is a fine substitute if you run into trouble.}
    %as I did in
    %\LuaLaTeX{} (see \https{github.com/reutenauer/polyglossia/issues/68}),
\item Try typesetting Japanese or Chinese with the \texttt{xeCJK} or
    \monobox{luatex-ja} packages.
\end{itemize}
\end{flushleft}

\chapter{When Good Type Goes Bad}

With luck, you're off to a solid start with \LaTeX.
But as with any complicated tool, you'll eventually run into trouble.
Here are some common problems and things you can try to fix them.

\section{Fixing overflow}

\mbox{When \LaTeX{} can't break a paragraph into well-spaced lines,
it gives up and overflows} into the margin.
You can sometimes remedy this with some ``emergency stretch''\quotekern.
If you add \texttt{\textbackslash emergencystretch=\allowbreak<width>}
to the preamble,
\LaTeX{} will try to set troublesome paragraphs a second time,
stretching or shrinking the total space in each line by up to the provided
width.\punckern\footnote{\LaTeX{} has pretty sane defaults for how much
it stretches and shrinks spacing.
You probably don't want to make \texttt{<width>} larger than an em or two.}
If that still doesn't help, tweak the wording of problematic paragraphs.
This can be frustrating, but the alternative is for \LaTeX{} to create spacing
that is too loose---where\quad
words\quad have\quad large\quad gaps\quad between\quad
them---or too tight, where\! words\! are\! awkwardly\! crammed\! together.

\section{Avoiding widows and orphans}

Good layouts avoid \introduce{widow} and \introduce{orphan}
(also called \introduce{club}) lines:
ones that get separated from the rest of their paragraph by a page boundary.
\LaTeX{} tries to prevent these, but its page-splitting algorithm
is much more primitive than its paragraph-splitting one.\punckern\footnote{%
This is because 1980s computers didn't have enough \acronym{ram} to do so.
Seriously---Knuth wrote at the time,
``The computer doesn't have enough high-speed memory capacity to remember the
contents of several pages,
so \TeX{} simply chooses each page break as best it can, by a process of
`local' rather than `global' optimization.\quotekern''\,\endnote{Knuth,
\textit{The \TeX book}, 110}}
You can make \LaTeX{} try harder to avoid orphans and widows with:
\begin{leftfigure}
\begin{lstlisting}
\widowpenalty=<penalty>
\clubpenalty=<penalty>
\end{lstlisting}
\end{leftfigure}
\verb|<penalty>| is a value between 0 and 10000.
When these values are maximized,
\LaTeX{} is never allowed to leave orphans or widows,
at \emph{any} cost.\punckern\footnote{When considering a given layout,
\LaTeX{} assigns penalties, or ``badness''\quotekern,
to anything that arguably makes a document look worse.
It chooses whichever layout it can find with the least badness.}
This can produce some really odd layouts,
so be sure to review your pages if you choose large penalties.

\section{Handling syntax errors}
If you confuse \LaTeX{}---say, by issuing commands that don't exist,
or forgetting to end an environment---it will print an
error message,\punckern\footnote{Usually this contains a succinct summary of
the problem and the number of the line(s) it occurred on. Occasionally,
\LaTeX{} gets \emph{really} confused and emits something so cryptic it gives
\cpp{} template errors a run for their money.
As you use \LaTeX, you'll start to get a feel for what sorts of
mistakes cause these rare, but enigmatic messages.}
then display an interactive prompt starting with \texttt{?}\,.
Here you can enter instructions for how to proceed.
Once upon a time, when computers were thousands of times slower and
\LaTeX{} took that much longer to re-run, this was more useful.
Today, we probably just want to quit,
then try again once we've fixed our document.
To exit the prompt, type \texttt{X}, then press Enter.
Better yet, you can tell \LaTeX{} to give up as soon as it finds trouble
by running your engine with the \monobox{-halt-on-error} flag:
\begin{leftfigure}
\begin{lstlisting}
$ lualatex -halt-on-error myDocument.tex
\end{lstlisting}
\end{leftfigure}


\appendix

\chapter{A Brief History of \texorpdfstring{\LaTeX}{LaTeX}}

\label{history}

Donald Knuth is celebrated among programmers as
a pioneer in computer science.
The man who coined the term \emph{analysis of algorithms},
Knuth is perhaps most famous for his ongoing magnum opus,
\textit{The~Art of Computer Programming}.

When the first volume of \acronym{taocp} was released in 1968,
it was printed the way most books had been since the turn of the century:
with \introduce{hot metal} type.
Letters were cast from molten lead,
then arranged into lines.
These lines were clamped together to form pages,
which were inked and pressed against paper.

By March of 1977, Knuth was ready for a second run of \acronym{taocp}, volume~2,
but he was horrified when he received the proofs.
Hot metal typesetting is expensive, complicated, and time-consuming,
so Knuth's publisher had replaced it with phototypesetting,
a process that projects characters' images onto film.
The new technology, while cheaper and faster,
didn't provide the quality Knuth
expected.\punckern\endnote{Knuth, \textit{Digital Typography} (Stanford, 1999), 3--5}

The average author would have resigned themselves to this change and moved on,
but Knuth took great pride in his books' appearances,
especially when it came to their mathematics.
Around this time, he also discovered the growing field of digital typesetting.
Inspired,
Knuth set off on one of the greatest yak shaves\footnote{Programmers
call seemingly-unrelated work needed to solve their main problem
``yak shaving''\quotekern. The phrase is thought to originate from an episode
of \textit{The Ren~\&~Stimpy Show}.\punckern\endnote{``yak shaving''\quotekern,
\textit{The Jargon File},
\href{http://www.catb.org/~esr/jargon/html/Y/yak-shaving.html}%
{\texttt{www.catb.org/\~{}esr/jargon/html/Y/yak-shaving.html}}}}
of all time.
For years, he paused work on his books to create his own
typesetting system.
When the dust settled in 1978, he had the first version of
\TeX.\punckern\footnote{The name ``\TeX{}'' comes from the Greek
{\fontspec[Scale=MatchLowercase]{NotoSerif-Medium}τέχνη},
meaning \introduce{art} or \introduce{craft}.\punckern\endnote{Knuth,
\textit{The \TeX book}, 1}}

It's hard to appreciate how much of a revolution \TeX{} was,
especially looking back from modern times, where anybody with a web browser
can be their own desktop publisher.
Adobe's \acronym{pdf} wouldn't exist for another decade, so Knuth
and his graduate students devised their own device-independent (document) format,
\acronym{dvi}.
Scalable fonts were uncommon, so he created \MF{} to rasterize glyphs
into dots on the page.
Perhaps most importantly, Knuth and his students designed algorithms
to automatically hyphenate and justify text into
beautifully-typeset paragraphs.\punckern\footnote{These same algorithms went
on to influence the ones Adobe uses in its software today.\punckern\endnote{%
Several sources (\http{www.tug.org/whatis.html},
\https{tug.org/interviews/thanh.html},
\http{www.typophile.com/node/34620})
mention \TeX's influence on the \textit{hz}-program by Peter Karow
and Hermann Zapf, thanks to via Knuth's collaborations with Zapf.
\textit{hz} was later acquired by Adobe and used
when creating InDesign's paragraph formatting systems.}}

\LaTeX{}, short for Lamport~\TeX{}, was later developed by Leslie Lamport
as a set of commands for common document layouts.
It was introduced in 1986 with his guide,
\textit{\LaTeX: A~Document Preparation System}.
Other typesetting systems based on \TeX{} also exist,
such as Con\TeX{}t.

Development continues today,
both in the form of user-provided packages for \TeX{} and \LaTeX{},
and as improvements to the \TeX{} typesetting program itself.
There are four versions, or \introduce{engines}:
\begin{description}
\item[\TeX] is the original system by Donald Knuth.
Knuth stopped adding features after version 3.0 in March~1990,
and all subsequent releases have contained only bug fixes.
With each release, the version number asymptotically approaches $\pi$
by adding an additional digit.
The most recent version, 3.14159265, came out in January~2014.

\item[pdf\TeX] is an extension of \TeX{} that provides direct \acronym{pdf}
    output (instead of \TeX's \acronym{dvi}),
    native support for PostScript
    and TrueType fonts,
    and micro-typographic features discussed in \chapref{microtype}.
    It was originally developed by
    Hàn Thế Thành
    as part of his PhD thesis
    for Masaryk University in Brno, Czech Republic.\punckern\endnote{%
    Hàn Thế Thành,
    \textit{Micro-typographic extensions to the \TeX{} typesetting system}
    (Masaryk University Brno, October 2000)}

\item[\XeTeX] is a further extension of \TeX{} that adds native support for
    Unicode and OpenType.
    It was originally developed by Jonathan Kew in the early 2000s,
    and gained full cross-platform support in 2007.\punckern\endnote{Jonathan Kew,
    ``\XeTeX{} Live''\quotekern, \textit{TUGboat} 29, no.~1 (2007)}

\item[\LuaTeX] is similar to \XeTeX{} in its native Unicode and modern font support.
    It also embeds the Lua scripting language into the engine,
    exposing an interface for package and document authors.
    It first appeared in 2007 and is developed by a core team of
    Hans Hagen, Hartmut Henkel, Taco Hoekwater,
    and Luigi Scarso.\punckern\endnote{\http{www.luatex.org}}
\end{description}

Building \TeX{} today is an\dots{} interesting endeavor.
When it was written in the late 1970s,
there were no large, well-documented, open-source projects for computer science
students to study,
so Knuth set out to make \TeX{} into one.
As part of this effort, \TeX{} was written in a style he calls
\introduce{literate programming}: opposite most programs---where
documentation is interspersed throughout the code---Knuth wrote \TeX{} as a book,
with the code inserted between paragraphs.
This mix of English and code is called \texttt{WEB}.\punckern\footnote{Knuth
also released a pair of companion programs named
\texttt{TANGLE} and \texttt{WEAVE}.
The former extracts the book---as \TeX, of course---and the latter
produces \TeX's Pascal source code.}

Unsurprisingly, most modern systems don't have good tooling for the late 1970s
dialect of Pascal that \TeX{} was written in,
so present-day distributions use another program,
\texttt{web2c}, to convert its \texttt{WEB} source into C code.
pdf\TeX{} and \XeTeX{} are built by combining the result with other C
and \cpp{} sources.
Instead of taking this complicated approach,
the \LuaTeX{} authors hand-translated Knuth's Pascal into C.
They have used the resulting code since 2009.\punckern\endnote{%
Taco Hoekwater, \textit{\LuaTeX{} says goodbye to Pascal}
(MAPS 39, Euro\TeX{} 2009),
\https{www.tug.org/TUGboat/tb30-3/tb96hoekwater-pascal.pdf}}


\setlength\parskip{0.55\baselineskip}
\setlength\parindent{0pt}

\chapter{Additional Resources}
\label{resources}

\section{For \texorpdfstring{\LaTeX}{LaTeX}}

As promised from the start, this book is incomplete.
To keep it short,
major \LaTeX{} features---like figures, captions, tables, graphics,
and bibliographies---haven't been discussed.
Use some of these resources to fill in the gaps:
\begin{leftfigure}
The \LaTeX{} Wikibook, at \https{en.wikibooks.org/wiki/LaTeX}

\textit{The Not So Short Introduction to \LaTeX}, \\
available at \https{www.ctan.org/tex-archive/info/lshort/english/}

The Share\LaTeX{} knowledge base, at \https{www.sharelatex.com/learn}

The \TeX{} Stack Exchange, at \https{tex.stackexchange.com/}
\end{leftfigure}

\section{For typography}

We've spent most of our time here focusing \emph{what} you can do with \LaTeX,
and little on \emph{how} you should use it to create well-designed documents.
Read on:
\begin{leftfigure}
\textit{Practical Typography}, by Matthew Butterick. \\
Available (for free!) at \https{practicaltypography.com}

\textit{Stop Stealing Sheep \& Find Out How Type Works}, by Erik Spiekermann

\textit{Thinking With Type}, by Ellen Lupton

\textit{Shaping Text}, by Jan Middendorp

\textit{The Elements of Typographic Style}, by Robert Bringhurst

\textit{Detail in Typography}, by Jost Hochuli
\end{leftfigure}

\backmatter

\setkomafont{chapter}{\Huge\itshape}

% Chicago Manual of Style, Notes & Bib style, ish.
% http://www.chicagomanualofstyle.org/tools_citationguide/citation-guide-1.html
\chapter{Notes}

% Endnotes _mostly_ works... with Koma Script, no less.
% I suppose I can only grumble a little about the hackery below.

% The \llap in \enoteformat throws the endnote number into the margin.
% To compensate, measure about the width of our widest endnote mark
% (see \makeenmark)...
\newlength{\enotewidth}
\settowidth{\enotewidth}{00.\enspace}

% We slapped down our own endnotes heading above with \chapter{Notes},
% so make the package do nothing when it tries.
\renewcommand\enoteheading{}
% Use our normal figures here instead of the superiors we marked endnotes with
% in the body text.
\renewcommand\makeenmark{\theenmark.\enspace}
% Normal size
\renewcommand\enotesize{\normalsize}
% TeX hackery lifted from endnotes.sty
% See https://www.tutorialspoint.com/tex_commands/llap.htm
% https://tex.stackexchange.com/q/22852/92465
\renewcommand\enoteformat{\leavevmode\llap{\makeenmark}}
% Indent the whole section by our measured amount much.
\begin{adjustwidth}{\enotewidth}{0pt}
\raggedright
\theendnotes
\end{adjustwidth}

% Redefine cleardoublepage so the Colophon doesn't demand a front page.
% From https://tex.stackexchange.com/a/24068/92465
{\let\cleardoublepage\clearpage \chapter{Colophon}}

This guide was typeset with \LuaLaTeX{} in Robert Slimbach's Garamond Premier.
His revival is based on roman type by
\otford{16}{th} century French
punchcutter Claude Garamond.
Italics are inspired by the work of Garamond's contemporary Robert Granjon.

Monospaced items are set in \texttt{Drive Mono},
designed by Elliott Amblard and Jérémie Hornus at Black Foundry.

Captions use
\textsf{\small Neue Haas Grotesk},
a Helvetica restoration by Christian Schwartz.
Other digitizations of the ubiquitous Swiss typeface are based on fonts made for
Linotype and phototypesetting machines,
resulting in digital versions with all the compromises and kludges from the
two previous generations of printing technology.
Schwartz based his work on Helvetica's original drawings,
producing a design faithful to the original cold metal type.

{\fontspec[Ligatures=TeX, Scale=MatchLowercase]{Futura-Boo}URW Futura}
makes a few guest appearances.
Designed by Paul Renner and first released in 1927,
Futura has found itself almost everywhere,
from advertising and political campaigns to the moon.
Douglas Thomas's recent history of the typeface,
{\fontspec[Ligatures=TeX, Scale=MatchLowercase]{Futura-Boo}Never Use Futura},
is a fantastic read.

Various bits of non-Latin text are set in
{\fontspec[Ligatures=TeX,Scale=MatchLowercase]{NotoSerif-Regular}Noto},
a type family by Google that covers \emph{every} glyph
in the Unicode standard.

Finally,
{\lm Latin Modern}---the OpenType version of Knuth's Computer Modern used throughout
the book---and
{\fontspec[Scale=MatchUppercase]{TeX Gyre Termes}\TeX{} Gyre Termes}---the
free alternative to Times Roman seen on page \pageref{typography}---are the
work of Grupa Użytkowników Systemu \TeX{}, the Polish \TeX{} Users' Group.
Overviews of these excellent projects can be found
on the \textsc{gust} website:\\
\http{www.gust.org.pl/projects/e-foundry/latin-modern} \\
\http{www.gust.org.pl/projects/e-foundry/tex-gyre}.


\end{document}
