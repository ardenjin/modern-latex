\chapter{Typography and You}
\label{typography}

Modern life is a parade of written language.
Wherever you go,
ads, apps, articles, essays, emails, and messages
shove text in your face.
But when you read that text, you see so much more than the authors'
verbiage.
Consciously or not, you notice the shapes and sizes of letters.
You notice how those letters are arranged into words,
how those words are arranged into paragraphs,
how those paragraphs are arranged onto pages and screens.
You notice \introduce{typography}.
% Smoke test: a line should be 2-3 alphabets wide
%\\ abcdefghijklmnopqrstuvwxyzabcdefghijklmnopqrstuvwxyzabcdefghijklmnopqrstuvwxyz
\begin{leftfigure}
\fontspec{TeX Gyre Termes}\fontsize{12bp}{24bp}\selectfont\raggedright
Typography is why these lines evoke memories of awful essays
you wrote in school.
Do many books look this way? Why not?
\end{leftfigure}
\medskip

\noindent Typography is why you might think of a shoe company that took
its name from a Greek goddess when you see this:
\begin{leftfigure}
\fontspec{FuturaCon-ExtBol}\Large JUST DO IT LATER.
\end{leftfigure}
With typography's help, two men landed their spaceship on a dusty alien plain.
Their ship carried a plaque, etched with the same shapes that adorned every
switch, knob, and readout inside. The plaque looked like this:
\begin{center}
\fontspec[Ligatures=TeX, Scale=MatchLowercase]{Futura-Med}
HERE MEN FROM THE PLANET EARTH \\
FIRST SET FOOT UPON THE MOON \\
JULY 1969, A.~D. \\
WE CAME IN PEACE FOR ALL MANKIND
\end{center}

Good typography isn't just art---it's function.
It draws readers in,
sets their expectations, and builds a subliminal brand for your
work.\punckern\endnote{Erik Spiekermann, ``Type is Visible Language''
(presented at Beyond Tellerrand, Düsseldorf, Germany, May 19--21, 2014),
\https{www.youtube.com/watch?v=ggQpDu63kk0}}
And if you care about any of this,
you you should try \LaTeX,\punckern\footnote{Pronounced ``lay-tech''
or ``lah-tech''}
a program for crafting written documents.
By carefully arranging subtle details,
it produces beautiful typography with little effort.
Modern versions can also leverage recent\footnote{By recent,
I mean ``from the mid-1990s''\quotekern, but web browsers and desktop publishing
software are only just starting to catch up.} advances in digital typesetting,
offering you the same tools used by professional graphic designers and
publishers.

\section{\texorpdfstring{\LaTeX}{LaTeX}?}

\LaTeX{} is an alternative to ``word processors'' like
Microsoft Word, Apple Pages, Google Docs,
and LibreOffice Writer.
These other applications operate on the principle of
\introduce{What You See Is What You Get}
\acronym{(wysiwyg)}, where what's on screen is the same
as what comes out of your printer.
\LaTeX{} is different. Here, documents are written as
``plain'' text files, using \introduce{markup} to specify
how the final result should look.
If you've done any web development, this is a similar
process---just as \acronym{html} and \acronym{css} describe
the page you want browsers to draw, your markup describes
the appearance of your document to \LaTeX.

\begin{samepage}
\begin{leftfigure}
\begin{lstlisting}
\LaTeX{} is an alternative to ``word processors'' like
Microsoft Word, Apple Pages, Google Docs,
and LibreOffice Writer.
These other applications operate on the principle of
\introduce{What You See Is What You Get}
\acronym{(wysiwyg)}, where what's on screen is the same
as what comes out of your printer.
\LaTeX{} is different. Here, documents are written as
``plain'' text files, using \introduce{markup} to specify
how the final result should look.
If you've done any web development, this is a similar
process---just as \acronym{html} and \acronym{css} describe
the page you want browsers to draw, your markup describes
the appearance of your document to \LaTeX.
\end{lstlisting}
\captionof{figure}{The \LaTeX{} markup for the paragraph above}
\end{leftfigure}
\end{samepage}

This might seem strange if you haven't worked with markup before,
but it comes with a few advantages:
\begin{enumerate}
\item You can handle your writing's content and its presentation separately.
    At the start of each document,
    you describe the design you want.
    \LaTeX{} takes it from there, consistently formatting your whole text.
    Compare this to a \acronym{wysiwyg} system,
    where you constantly deal with appearances
    as you write.
    If you changed the look of a caption,
    were you sure to find all the other captions and do the
    same?
    If the program formats something in a way you don't like,
    is it hard to fix?%\footnote{I spent far too much of my childhood
    %fighting with Word about how it wrapped text around images.}

\item You can define your own commands, then tweak them to instantly adjust
    every place they're used.
    For example, the \verb|\introduce| and \verb|\acronym| commands
    from the example paragraph are my own creations.
    The former \introduce{italicizes} text, and the latter sets words in
    \acronym{small caps} with a bit of extra
    \mbox{\textsc{\addfontfeature{LetterSpace=15}letterspacing}} so the characters
    don't look \textsc{\addfontfeature{LetterSpace=-10}too crowded}.
    If I decide tomorrow that I would rather introduce new terms
    \textbf{\itshape with this look}, or that acronyms should look
    {\small\addfontfeature{LetterSpace=6} LIKE THIS},
    I just change the two lines that define those commands.
    Every spot in this book that uses them is immediately updated.

\item Being able to save the document as plain text also has benefits:
    \begin{itemize}
    \item It can be read and understood with any text editor.
    \item Structure is immediately visible
        and simple to replicate.\punckern\footnote{Compare this to
        \acronym{wysiwyg} systems, where it's not always obvious
        how certain formatting was produced or how to replicate it.}
    \item Content is easily generated by scripts and programs.
    \item Changes can be tracked with standard version control software,
        like Git or Mercurial.
    \end{itemize}
\end{enumerate}

\section{Another guide?}

You might wonder why the world needs another guide for \LaTeX{}.
After all, it's existed for decades.
A quick Amazon search finds nearly a dozen books on the topic.
There are plenty of great resources online.

Unfortunately, most of the introductions you will find have two fatal flaws:
they are long, and they are old.
A 200+ page book seems daunting to anybody just trying to learn the basics,
and age matters because of how much typesetting has changed since 1986.
When \LaTeX{} was first released that year, none of the publishing technologies
we use today existed.
Adobe wouldn't debut their Portable Document Format for seven more years,
and desktop publishing was a fledgling curiosity.
This shows---badly---in most \LaTeX{} guides.
If you look for instructions to change your document's font,
you'll get swamped with bespoke nonsense.\punckern\footnote{%
Take any criticisms that you find here with a grain of
salt. After all, the fact that all of the technology around \LaTeX{} became
obsolete---multiple times---is a testament to its staying power.}

The good news is that  \LaTeX{} has improved by leaps and bounds in recent years.
It's time for a guide that doesn't weigh you down with decades of legacy
or try (in vain) to be a comprehensive reference.
After all, you're a smart, resourceful individual who sling a search engine.
This book will:

\begin{enumerate}
\item Teach you the fundamentals of \LaTeX.
\item Point you to places where you can learn more.
\item Show you how to use modern typesetting technologies and techniques,
    like OpenType and microtypography.
\item End promptly thereafter.
\end{enumerate}
\vspace{\baselineskip}

\noindent Let's begin.
