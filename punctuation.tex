\chapter{Punctuation}
\label{punctuation}

You would much rather encounter a panda that eats
shoots and leaves than one that eats, shoots,
and leaves.\punckern\endnote{Lynne Truss,
\textit{Eats, Shoots \& Leaves} (New York, 2003)}
Punctuation is a vital part of writing,
and there's more to it than your keyboard suggests.

\section{Quotation marks}

\LaTeX{} doesn't automatically convert \texttt{"}straight\texttt{"} quotes
into correctly-facing ``curly'' ones:
\begin{leftfigure}
\begin{lstlisting}
"This isn't right."
\end{lstlisting}
\end{leftfigure}
will get you
\begin{leftfigure}
\lm%
"This isn't right."
\end{leftfigure}
Instead, use \texttt{\bfseries `} for opening quotes and \texttt{\bfseries '} for closing
quotes.\punckern\footnote{If your keyboard happens to have keys for
``curly'' quotes (\,\texttt{“}\,\texttt{”}\,), feel free to use those instead!
Also, don't use \texttt{"} for closing double quotes.
Not only does \texttt{``example"} look a bit unbalanced,
but \texttt{"} is used as a formatting command when typesetting certain
languages, like German. (See \chapref{i18n} for more on international
typesetting.)}
\begin{leftfigure}
\begin{lstlisting}
``It depends on what the meaning of the word `is' is.''
\end{lstlisting}
\end{leftfigure}
quotes a former \acronym{us} president as,
\begin{leftfigure}
\lm%
``It depends on what the meaning of the word `is' is.''
\end{leftfigure}

\section{Hyphens and dashes}

Though they look similar,
hyphens (\,-\,), en dashes (\,--\,),
em dashes (\,---\,), and minus signs (\,$-$\,)
serve different purposes.
\begin{description}
\item[Hyphens] have a few applications:\endnote{Matthew Butterick,
    ``Hyphens and dashes''\quotekern,
    \textit{Practical Typography},
    \https{practicaltypography.com/hyphens-and-dashes.html}}
    \begin{itemize}[leftmargin=*]
    \item They allow words to be split across the end of one line and the
        start of the next.
        \LaTeX{} usually handles this automatically.
    \item Compound words like \emph{long-range} and \emph{field-effect}
        use hyphens.
    \item They are used in phrasal adjectives.
        If I ask for ``five dollar bills''\punckern,
        do I want five \$1 bills, or several \$5 bills?
        It's clearer that I want the latter when typeset as
        \emph{five-dollar bills}.
    \end{itemize}
    Unsurprisingly, you get one by typing the hyphen character (\,\texttt{-}\,).

\item[En dashes] are for ranges such as ``pages 4--12''\quotekern,
    and connected words, like ``the \mbox{US--Canada} border''\quotekern.
    \LaTeX{} places one wherever it sees two adjacent hyphens
    (\,\texttt{--}\,).

\item[Em dashes] separate clauses of a sentence.
    Other punctuation---like parenthesis and commas---play a similar
    role.
    Em dashes are typeset with three hyphens (\,\texttt{---}\,).

\item[Minus signs] are for negative quantities and
    mathematical expressions.
    They are similar in length to an en dash,
    but sit at a different height.
    Minus signs are set with \verb|\textminus|,
    or with the hyphen character
    when in math environments (see \chapref{math}).
\end{description}

\section{Ellipses}

A set of three dots used to indicate a pause or omission is called an
\introduce{ellipsis}.
It is set with \verb|\dots|.
\begin{leftfigure}
\begin{lstlisting}
I'm\dots{} not sure.
\end{lstlisting}
\end{leftfigure}
becomes
\begin{leftfigure}
\lm%
I'm\dots{} not sure.
\end{leftfigure}
Ellipses have different spacing than consecutive periods.
Don't use the latter as a poor substitute for the former.

\section{Spacing}

As we discovered in \chapref{hello},
\LaTeX{} inserts additional space between periods and whatever
follows them---presumably the start of the next sentence.
This isn't always what we want!
Consider honorifics like Mr.\ and Ms., for example.
In these situations, we also need to prevent \LaTeX{} from starting a
new line after the period.
This calls for a \introduce{non-breaking space}, which we set with a tilde.
\begin{leftfigure}
\begin{lstlisting}
Please call Ms.~Shrdlu.
\end{lstlisting}
\end{leftfigure}
produces proper spacing:
\begin{leftfigure}
\lm%
Please call Ms.~Shrdlu.
\end{leftfigure}

In other occasions, like when we abbreviate units of
measurement,\punckern\footnote{There are also dedicated packages for doing so,
like \texttt{siunitx}.}
we want thinner spaces than our usual inter-word ones.
For these, we use \verb|\,|\,:
\begin{leftfigure}
\begin{lstlisting}
Launch in 2\,h 10\,m.
\end{lstlisting}
\end{leftfigure}
announces
\begin{leftfigure}
\lm%
Launch in 2\,h 10\,m.
\end{leftfigure}

\section{What next?}
\begin{itemize}
\item Learn more commands for spacing, such as \verb|\:|, \verb|\;|,
    \verb|\enspace|, and \verb|\quad|.
\item Nest quotations, e.g.,
    \enquote{She exclaimed, \enquote{I can't believe it!}}
    more easily with  \texttt{csquotes} package's \verb|\enquote| command.
\item Discover the typographical origins of terms like \introduce{en},
    \introduce{em}, and \introduce{quad}.
\item Familiarize yourself with the difference between \texttt{/} and
    \verb|\slash|.
\item Add hyphenations for uncommon words using \verb|\hyphenate|
    or \verb|\-|\,.\punckern\footnote{\LaTeX{} usually does a good
    job of automatically hyphenating words, based on a dictionary of patterns
    stored for each language. You should rarely need these commands.}
\end{itemize}
